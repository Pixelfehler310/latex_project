% ### 2.1 Human-Centered Design (HCD) und Usability
\subsection{Human-Centered Design (HCD) und Usability}\label{subsec:hcd-usability}

% #### 2.1.1 Kognitive Aspekte der Interaktion (Psychologie, DOET und CLT)
\subsubsection{Kognitive Aspekte der Interaktion (Psychologie, DOET und CLT)}\label{subsubsec:kognitive-aspekte}

% * **Inhalt:**
% >
% > - **Problem**: Nutzersysteme mit schlechter Usability führen zu hoher kognitiver Last und Fehleranfälligkeit. -> Beispiel!
% > - **Norman's Konzepte**: Affordances, Signifiers, Mapping, Feedback, Constraints, Conceptual Model.
% > - **Cognitive Load Theory**: Extrinsische Last (durch schlechtes Tool) vs. Intrinsische Last (Aufgabe). Ziel: Reduktion der extrinsischen Last durch den visuellen Editor.
% > - **Gulf of Execution/Evaluation** (Norman): Die Kluft zwischen Nutzerziel und Systembedienung verringern.
% > - **Definitionen Usability & User Experience** (ISO 9241-11, ISO 9241-210 und weitere finden)
% > - Reihenfolge Variabel!

% todo The invisible Computer durchsuchen, um als zweitbasis quelle zu nutzen

Softwaresysteme werden in der heutigen Welt immer komplexer. Um die Kunden zu überzeugen werden immer mehr Features und Funktionalitäten integriert. Dies führt jedoch oft dazu, dass die Bedienung dieser Systeme für die Nutzer zunehmend schwieriger und unübersichtlicher wird. % src: The Invisible Computer, Norman 1998, p.79
Diese Faktoren werden von der Digitalisierung und der zunehmenden Vernetzung und Automatisierung weiter verstärkt. % nosrc

Komplexer werdende Systeme sind eine Herausforderung für die Nutzer. Sie müssen sich mehr Abläufe, Funktionen, Bedienmöglichkeiten und Shortcuts merken. Das Problem dabei ist, dass der Mensch nur eine begrenzte Kapazität im Kurzzeitgedächtnis, oder auch Arbeitsgedächtnis genannt, hat, um Informationen zu verarbeiten und zu speichern. % verify in: Cognitive Load Theory, Sweller 1988
% src 2: DOET p. 100
1988 hat Sweller die Cognitive Load Theory (CLT) entwickelt, die sich mit der kognitiven Last beschäftigt, die bei der Informationsverarbeitung im Arbeitsgedächtnis entsteht. % src: Cognitive Load Theory, Sweller 1988
Die kognitive Last bezeichnet die Menge an mentaler Anstrengung, die erforderlich ist, um eine bestimmte Aufgabe zu bewältigen. % nosrc
Wenn die kognitive Last zu hoch wird und das Arbeitsgedächtnis des Nutzers überfordert ist, steigt die Wahrscheinlichkeit für Fehler bei der Bedienung des Systems. Nutzer übersehen wichtige Funktionen, vergessen Abläufe, verklicken sich, oder treffen falsche Entscheidungen. Dies führt zu Frustration, Unzufriedenheit und letztlich zur Ablehnung des Systems. % nosrc

Die CLT unterscheidet zwischen drei Arten von kognitiver Last: der intrinsischen Last, der extrinsischen Last und der germanen Last. % src: Cognitive Load Theory, Sweller 1988

Die intrinsische Last bezieht sich auf die Komplexität der zu erlernenden Aufgabe selbst. % src: Cognitive Load in UX Design 
% addsrc
Ist eine Aufgabe komplexer, erfordert sie mehr kognitive Ressourcen, um sie zu verstehen und zu bewältigen. % src: Cognitive Load in UX Design 

Die extrinsische Last hingegen ist die Last, die durch überflüssige, für das Systemverständnis irrelevante, Informationen entsteht, wie z.B. eine unübersichtliche Benutzeroberfläche, verwirrende Diagramme oder unnötige Ablenkungen. % src: Cognitive Load in UX Design 
% addsrc

Die germane Last bezieht sich auf die mentale Anstrengung, die der Nutzer aufwendet, um neues Wissen zu konstruieren und in das bestehende Wissen zu integrieren. % src Cognitive Load in UX Design
% addsrc

Die intrinsische Last ist oft unvermeidbar, da sie von der Natur der Aufgabe abhängt. Die germanische Last ist wünschenswert, da sie das Lernen und die Wissenskonstruktion fördert. Die extrinsische Last aber ist vermeidbar und kann durch eine gute Gestaltung des Systems reduziert werden. Ziel ist es, die extrinsische Last so gering wie möglich zu halten, um die kognitive Kapazität des Nutzers für die eigentliche Aufgabe freizusetzen und die Wahrscheinlichkeit von Fehlern zu minimieren. % src: Cognitive Load in UX Design
% addsrc
Da die extrinsische Last die einzige zu minimierende Last ist, ist in den folgenden Abschnitten die Reduzierung der kognitiven Last meist als Reduzierung der extrinsischen Last zu verstehen. % refine

Um die kognitive Anstrengung bei der Nutzung eines Systems zu senken muss man verstehen, durch welche Probleme die kognitive Anstrengung entsteht. Laut Hutchins, Hollan und Norman gibt es zwei Hauptprobleme, die bei der Interaktion zwischen Mensch und System auftreten können und kognitive Last verursachen: den Gulf of Execution und den Gulf of Evaluation.
% src: Direct Manipulation Interfaces, Hutchins, Hollan, Norman 1985, p. 318 (aus Human Computer Interaction)

Der Gulf of Execution beschreibt die Informationslücke die der Nutzer überwinden muss um herauszufinden wie das System funktioniert und wie er sein Ziel erreichen kann. % src: The Design of Everyday Things, Norman p.53 % src2: Direct Manipulation Interfaces, Hutchins, Hollan, Norman 1985

Der Gulf of Evaluation ist die Informationslücke, die der Nutzer überwinden muss um zu verstehen was im System passiert ist, nachdem er eine Aktion ausgeführt hat, und ob ihm das geholfen hat sein Ziel zu erreichen. % src: The Design of Everyday Things, Norman p.53
Das Ziel eines gut gestalteten Systems ist es, diese beiden Informationslücken so klein wie möglich zu halten, indem es dem Nutzer klare Signale und Rückmeldungen gibt, die ihm helfen, das System zu verstehen und effektiv zu nutzen. % src: The Design of Everyday Things, Norman p.53

Donald (Don) Norman, Professor für Kognitionswissenschaften und Informatik, hat in seinen Büchern The Design of Everyday Things und The Invisible Computer die wesentlichen psychologischen Vorgänge im menschlichen Gehirn bei der Interaktion von Objekten und Systemen analysiert und Ansätze vorgestellt die Nutzbarkeit der Systeme zu verbessern und die kognitive Last für die Nutzer zu reduzieren. 
Norman beschreibt 6 psychologische Konzepte, die bei der Gestaltung von benutzerfreundlichen Systemen berücksichtigt werden sollten.
Dazu gehören unter anderem Affordances, Signifiers, Mapping, Feedback, Constraints und das Conceptual Model. % src: The Design of Everyday Things, Norman 2013; The Invisible Computer, Norman 1998

% TODO Discoverability noch erklären ??

Affordances sind Eigenschaften eines Systems, die Repräsentieren, was der User er tun kann. Sie definieren die Interaktionsangebote. % src: DOET p 27
Interaktionsangebote sind nicht immer offensichtlich. Eine Affordance bedeutet nicht automatisch, dass der Nutzer sie wahrnimmt. Bleibt die Interaktionsmöglichkeit unklar, kann sie nicht genutzt werden. Hier kommen Signifier ins Spiel. Signifier sind Hinweise oder Markierungen, die dem Nutzer die Interaktionsmöglichkeit, sowie deren Ausführung verdeutlichen. % src: DOET p 27-28
Sie sollen dem Nutzer kommunizieren, was der Zweck, der Aufbau und die Funktionsweise eines Objekts ist. % src: DOET p 30
Ein Signifier kann zum Beispiel ein Label auf einem Knopf sein, das suggeriert, was passiert, wenn dieser gedrückt wird, oder der Cursor der sich zum Pointer ändert, wenn er über einem Link hovert.
Eng verwandt mit den Signifiers sind Constraints. Anstatt zu zeigen was möglich ist, zeigen Constraints dem Nutzer was explizit nicht möglich ist.
Constraints begrenzen die möglichen Interaktionen, um Fehler zu vermeiden und die Bedienung zu leiten. Norman unterscheidet dabei zwischen physischen, kulturellen, semantischen und logischen Constraints. % src: DOET p p 128
Im Kontext von grafischen Benutzeroberflächen sind vor allem logische Constraints von Bedeutung. Sie nutzen das logische Verständnis des Nutzers über die Funktionsweise des Systems, um nur sinnvolle Aktionen zuzulassen. % src: DOET p 133
Ein klassisches Beispiel ist ein ausgegrauter (deaktivierter) Button in einem Formular, solange Pflichtfelder nicht ausgefüllt sind. Der Nutzer wird so aktiv daran gehindert, unvollständige Daten abzusenden, die zu Fehlern führen würden.

Das vierte Konzept ist das Mapping. Mapping beschreibt die Beziehung zwischen den Steuerelementen eines Systems und den Aktionen, die sie auslösen. % src: DOET p 37
Gutes Mapping bedeutet, dass die Anordnung der Steuerelemente intuitiv und logisch ist, sodass der User leicht verstehen kann, welche Steuerung welche Aktion auslöst. % src: DOET p 38

% todo Beispiel für Mapping

Feedback ist ein weiteres wichtiges Konzept. Es beschreibt die Rückmeldung, die ein System dem Nutzer gibt, nachdem eine Aktion ausgeführt wurde. % src: DOET p 40
Feedback informiert den Nutzer über den aktuellen Zustand des Systems oder die Ergebnisse seiner Aktionen. % src: DOET p 40

Feedback ist entscheidend, damit der User versteht, was im System passiert, und um ihm Sicherheit bei der Interaktion zu geben. Dauert Feedback zu lange oder fehlt es ganz, kann der User verwirrt werden und Fehler machen. % src: DOET p 41
Außerdem muss Feedback geplant und priorisiert werden, da auch zu viel oder unnötiges Feedback den User überfordern, ablenken und verwirren kann. % src: DOET p 41 
Dies kann in verschiedenen Formen geschehen, zum Beispiel durch visuelle Hinweise, akustische Signale oder haptische Rückmeldungen. % nosrc
Mit Feedback wird der Gulf of Evaluation verringert, da der Nutzer schneller und einfacher verstehen kann, was im System passiert ist und ob seine Aktionen erfolgreich waren. % nosrc

Die Kombination von Affordances, Signifiers, Constraints und Mapping ergibt ein Feedforward-System. Feedforward ist genau das Gegenteil von Feedback. Anstatt, dass dem User gezeigt oder erklärt wird, was er getan hat, nachdem er es getan hat, wird ihm gezeigt, was er tun kann und was die Resultate der möglichen Aktionen sind, bevor er sie tätigt. Dadurch wird der Gulf of Execution verkleinert, da der Nutzer weniger kognitive Anstrengung aufwenden muss, um herauszufinden, wie das System funktioniert und wie er sein Ziel erreichen kann. % src: DOET, p. 82,83
% TODO Sätze ein bisschen sortieren.

Das sechste Konzept ist das Conceptual Model. Ein Conceptual Model ist eine mentale Repräsentation, die der User von einem System entwickelt, basierend auf seinen Erfahrungen, seinem Wissen und den Informationen, die das System bereitstellt, über die Funktionsweise des Systems % src: DOET p 42   
% todo Satz verbessern
Oft entstehen diese Modelle durch das Lernen des Systems durch Interaktion und Ausprobieren, durch den Informationstransfer von Person zu Person, oder durch das Lesen von Dokumentationen oder ähnlichen Systembeschreibungen. % src: DOET p 43
Das Conceptual Model ist eng verbunden mit der germanen Last der CLT, da es das gelernte Wissen des Nutzers über das System repräsentiert und ihm hilft, neue Informationen zu integrieren. % refine
Außerdem hilft das Conceptual Model sowohl den Gulf of Execution, als auch den Gulf of Evaluation zu überbrücken, da das Systemverständnis, dass das Conceptual Model darstellt die Ursprungsprobleme löst. % src: Direct Manipulation Interfaces, Hutchins, Hollan, Norman 1985, p. 318
Denn wenn der User bereits weiß wie er das System bedienen muss, kann es nicht mehr zu Problemen führen.


% todo Beispiele für die psychologischen Konzepte einfügen (Affordances, Signifiers, Mapping, Feedback, Constraints, Conceptual Model) (jedes einzeln, direkt beim inhaltlichen Abschnitt)

% todo Feed-Forward noch erklären ??

% #### 2.1.2 Gestaltungsprinzipien für interaktive Systeme
\subsubsection{Gestaltungsprinzipien für interaktive Systeme}\label{subsubsec:gestaltungsprinzipien}

% * **Inhalt:**
% > - **Aus 2.1.1 ableitbare Designprinzipien**
% > - **Definitionen von Usability & UX** (ISO 9241-11, ISO 9241-210 und weitere finden)
% > - **kognitive zusammenfassung der wichtigsten Heuristiken, mit Fokus auf ISO NORM 9241:210:2010**: (Additiv und als BOnus quellen: Norman, Shneiderman, Nielsen, Honeycomb)
% > -  Nutzerführung (affordances, signifiers, constraints, mapping, feedback, conceptual model, feedforward)
% > - **Konsistenz**: Erwartungskonformität.
% >   Dialoggestaltungs Prinzipien (9241-110)
% > - explizit NICHT hier drin: Direkte und indirekte Manipulation und WYSIWYG (kommt in 2.3.1)

% todo Übergänge zwisdchen den Sätzen runder machen, ISt so ein stotternder Text

Aus den psychologischen Grundgedanken haben sich in der Softwareentwicklung Kernbegriffe herauskristallisiert. Die ISO Norm 9241 definiert Usability als "das Ausmaß, in dem ein System, Produkt oder Service von bestimmten Benutzern verwendet werden kann, um in einem bestimmten Nutzungskontext bestimmte Ziele effektiv, effizient und zufriedenstellend zu erreichen." % src: ISO 9241-210
% ! Direct quote (translated)
User Experience (UX) wird in der ISO 9241-210 als "die Wahrnehmungen und Reaktionen einer Person, die sich aus der Nutzung oder der erwarteten Nutzung eines Produkts, Systems oder Dienstes ergeben" definiert. % src: ISO 9241-210
% ! Direct quote (translated)

Das Ziel für interaktive Systeme ist es, eine hohe Usability und eine positive User Experience zu gewährleisten. % nosrc

Dazu wurden in ISO 9241-110 sieben Gestaltungsprinzipien entwickelt, die auf den zuvor beschriebenen psychologischen Konzepten basieren: % refine echte quelle holen wenn möglich (ISO 110)

Das erste Prinzip ist die Aufgabenangemessenheit. Es besagt, dass ein System Nutzende bei der Erreichung ihrer spezifischen Arbeitsziele effektiv und effizient unterstützen soll. Das kann durch Prozessoptimierung, Automatisierung oder die Vorauswahl von Standardeinstellungen erreicht werden, um den Aufwand für den Nutzer zu minimieren. % src: Dialogprinzipien wurden überarbeitet (ProContext, 2020), Thomas Geis

Eng damit verbunden ist die Selbstbeschreibungsfähigkeit, die sicherstellt, dass dem Nutzer zu jeder Zeit klar ist, in welchem Zustand sich das System befindet und welche Schritte als nächstes erwartet werden. Es sollte selbsterklärend sein. Dies korreliert direkt mit dem Einsatz von Signifiers und Feedback zum Bilden des Conceptual Models der Nutzer. So wird die Bedienung des Systems transparenter und verständlicher. % src: Dialogprinzipien wurden überarbeitet (ProContext, 2020), Thomas Geis

Das dritte Prinzip, die Erwartungskonformität, verlangt, dass sich das System konsistent verhalten und dem erwarteten Verhalten der Nutzer entsprechen sollte. % src: Dialogprinzipien wurden überarbeitet (ProContext, 2020), Thomas Geis
Wenn das Mapping und die Abläufe den Erfahrungen aus der physischen Welt oder bekannten Anwendungen gleichen, finden sich Nutzer schneller zurecht. % nosrc

Wenn das System den Nutzer beim Entdecken, Ausprobieren und Wiedererkennen von Bedienfunktionen unterstützt, und auch dazu motiviert, ohne negative Auswirkungen hervorzurufen entspricht es dem Prinzip der Erlernbarkeit. % src: Dialogprinzipien wurden überarbeitet (ProContext, 2020), Thomas Geis

Im System muss der Nutzer jederzeit die Kontrolle über das System haben und in der Lage sein, Prozesse zu starten, zu unterbrechen oder in ihrer Geschwindigkeit anzupassen. Zudem muss es flexibel sein, das heißt es passt sich den Gewohnheiten des Nutzers an und nicht umgekehrt. Außerdem muss es individualisierbar sein. Der Nutzer kann das System an seine Präferenzen anpassen. Dazu gehören zum Beispiel Accessibility-Einstellungen wie ein Hoher-Kontrast Modus, ein Farbenblindheits-Modus, aber auch der Wechsel zwischen Light- und Dark-Mode. Diese drei Ausprägungen sind zusammengefasst unter dem Prinzip der Steuerbarkeit. % src: Dialogprinzipien wurden überarbeitet (ProContext, 2020), Thomas Geis

Robustheit gegen Benutzungsfehler ist das sechste Prinzip. Es besagt, dass das System so gestaltet sein muss, dass es Fehlbedienungen möglichst vermeidet, und wenn sie doch auftreten, diese abgefangen werden und es dem Nutzer ermöglicht wird, sie zu beheben. % src s.o

Das letzte Interaktionsprinzip ist die Benutzerbindung. Es fordert, dass das System den Nutzer motiviert und ihm ein positives Nutzungserlebnis bietet, um eine langfristige Nutzung zu fördern. Dazu gehören Aspekte wie ansprechendes Design, Personalisierungsmöglichkeiten und die Berücksichtigung emotionaler Faktoren. % src s.o
% addsrc

% #### 2.1.3 Human Centered Design (HCD) Prinzipien
\subsubsection{Human Centered Design (HCD) Prinzipien}\label{subsubsec:hcd-prinzipien}

% * **Inhalt:**
%  > Der HCD Prozess mit seinen vier Phasen (Verstehen und Spezifizieren des Nutzungskontextes, Spezifizieren der Nutzungsanforderungen, Gestalten der Lösung, Evaluieren der Lösung) wird vorgestellt. Der Fokus liegt dabei auf den Prinzipien des nutzerzentrierten Designs, die in jeder Phase angewendet werden sollten, um sicherzustellen, dass die Bedürfnisse und Anforderungen der Nutzer im Mittelpunkt stehen.
% > Nach Iso Norm und ggf weitere quellen (Norman, Shneiderman)

% todo mit Normans HCD Vergleichen (und ggf. noch andere HCD Definitionen finden)

Die Interaktionsprinzipien und die psychologischen Konzepte bilden die Grundlage für den Human-Centered Design (HCD) Prozess. % nosrc

Human-Centered Design (HCD) ist ein Ansatz zur Gestaltung und Entwicklung interaktiver Systeme, bei dem der Mensch in den Mittelpunkt gestellt wird. Das Ziel ist es, die Bedienbarkeit der Systeme schon im Entwicklungsprozess zu verbessern, indem Wissen über die Nutzer, ihre Aufgaben und die Umgebung systematisch analysiert und einbezogen wird. % src: ISO 9241-210, p. 2, 2.7
Das Ergebnis eines auf HCD basierenden Entwicklungsprozesses sind bestenfalls Systeme, die eine hohe Usability aufweisen, also Systeme, die effektiv, effizient und zufriedenstellend von den Nutzern verwendet werden können. % src: ISO 9241-210, p. 3, 2.13

Gute Usability und die Nutzung von HCD-Prinzipien können helfen die Akzeptanz und Zufriedenheit der Nutzer zu erhöhen, die Produktivität zu steigern und Fehler zu reduzieren. Außerdem können sie die Trainings und Supportkosten senken, da Systeme intuitiver und leichter zu erlernen sind. Weniger Frustration bei der Nutzung durch gutes Design führt zudem zu einer positiveren Wahrnehmung des Produkts und der Marke. % nosrc

Die ISO Norm 9241-210:2010 beschreibt sechs Grundregeln des Human-Centered-Design-Prozesses: % src: ISO 9241-210, p. 5
\begin{enumerate}
    \item Der Entwurf basiert auf einem klaren Verständnis der Nutzer, deren Aufgaben und der Nutzungsumgebung.
    \item Die Nutzer werden aktiv in den Entwurfs- und Entwicklungsprozess einbezogen.
    \item Der Entwurf wird durch benutzerzentrierte Evaluation getrieben und verfeinert.
    \item Der Entwurf wird iterativ entwickelt.
    \item Alle Aspekte der User Experience werden berücksichtigt.
    \item Das Design wird von einem Team erstellt, das diversifizierte Fähigkeiten und Perspektiven mitbringt. % todo den Satz kann man verbessern % todo ggf ausführen welche Fähigkeiten es braucht
\end{enumerate}

% todo vier Hauptaktivitäten des HCD Prozesses erklären (Verstehen und Spezifizieren des Nutzungskontextes, Spezifizieren der Nutzungsanforderungen, Gestalten der Lösung, Evaluieren der Lösung) % src: ISO 9241-210, p. 6

% Ein beispielhafter HCD-Prozess könnte in etwa so aussehen: % todo Beispiel ausfeilen
% Zunächst muss man analysieren und verstehen, wer die Nutzer sind, welche Aufgaben sie erledigen müssen und in welchem Kontext sie das System verwenden. Basierend auf diesem Verständnis werden Anforderungen definiert, die das Design leiten.
% Anschließend werden erste Entwürfe entwickelt, die diese Anforderungen erfüllen. Der frühe Entwurf wird dann bereits an die User kommuniziert um Feedback einzuholen. Dieser Schritt nennt sich Validierung des Designs. Das frühe Einbinden der Nutzer in den Prozess ermöglicht es, aufkommende Probleme frühzeitig erkennen und beheben zu können.
% Mithilfe des User-Feedbacks wird das Produkt dann angepasst und verbessert. 

% fig Prozessbeispiel

Diese Prinzipien bedingen methodisch meist ein iteratives Vorgehen in vier Phasen: Verstehen des Nutzungskontextes, Spezifizieren der Anforderungen, Entwerfen von Designlösungen und Evaluieren gegen die Anforderungen. % src: ISO 9241-210, p. 10-19
Diese Phasen werden mehrfach durchlaufen und so wird das System kontinuierlich besser. Aufgrund dieser iterativen Struktur sind Agile Entwicklungsmethoden gut geeignet, um HCD-Prinzipien umzusetzen. 

Da im späteren Verlauf der Arbeit jedoch nur eine abgewandelte Form des HCD-Prozesses angewendet wird, wird an dieser Stelle auf eine detaillierte Beschreibung der einzelnen Phasen verzichtet.

Die Evaluation der Usability und der User Experience ist ein wichtiger Bestandteil des HCD-Prozesses. Doch um bewerten zu können, ob ein System eine gute User Experience oder Usability aufweist, muss man zunächst verstehen, was diese Begriffe bedeuten und wie sie gemessen werden können. 

% #### 2.1.4 Methoden der Usability-Evaluation
\subsubsection{Methoden der Usability-Evaluation}\label{subsubsec:usability-evaluation}

% * **Inhalt:**
% > - **UX Honeycomb** (Peter Morville): Useful, Usable, Desirable. (WICHTIG für die spätere Nutzwertanalyse, liefert die Kategorien)
% > - **Heuristische Evaluation** (Nielsen).
% > - **Usability Testing** (Nutzer-basierte Evaluation).
% > - **Qualitative vs. Quantitative Methoden**.
% > - **Ggf weitere Methoden wenn sinnvoll**

% user based testing und expert based testing sind auch in ISO 9241-210 beschrieben

