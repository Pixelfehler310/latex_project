% 2.3 Human-Centered Design (HCD) und Usability
\subsection{Human-Centered Design (HCD) und Usability}\label{subsec:hcd-usability}

% Einleitung:
% - Definition HCD (ISO 9241-210): Prozess, der den Menschen in den Mittelpunkt stellt.
% - Ziel: Gebrauchstaugliche (usable) und nützliche (useful) Systeme schaffen.
% - Abgrenzung zu Usability: HCD ist der Prozess, Usability ein Qualitätsmerkmal.

\subsubsection{Gestaltungsprinzipien für interaktive Systeme}\label{subsubsec:gestaltungsprinzipien}

% Hier die wichtigsten Prinzipien als Grundlage für gutes Interaktionsdesign vorstellen.

% \paragraph{Grundlegende Designprinzipien}
% - Sichtbarkeit (Visibility): Was kann ich tun?
% - Rückmeldung (Feedback): Was ist passiert?
% - Konsistenz (Consistency): Gleiche Dinge sollten gleich aussehen und funktionieren.
% - Affordances & Signifiers: Was ein Objekt kann (Affordance) und wie das kommuniziert wird (Signifier).

% \paragraph{Konzepte für Editoren und visuelle Werkzeuge}
% - Direkte Manipulation (Shneiderman): Gefühl der direkten Kontrolle über Objekte.
% - WYSIWYG (What You See Is What You Get): Die Darstellung im Editor entspricht dem Endergebnis.

% \paragraph{Heuristiken als praktische Leitlinien}
% - Nielsen's 10 Usability Heuristics als anerkannter Standard.
% - Shneiderman's "Golden Rules of Interface Design".

\subsubsection{Methoden der Usability-Evaluation}\label{subsubsec:usability-methoden}

% Überblick über Methoden, um die Usability eines Systems zu bewerten.
% Unterscheidung: Formative (während der Entwicklung) vs. summative (am Ende) Evaluation.

% \paragraph{Qualitative Methoden (Verstehen des "Warum")}
% - Thinking Aloud: Nutzer verbalisieren ihre Gedanken bei der Nutzung.
% - Cognitive Walkthrough: Experten simulieren den Weg eines Nutzers durch eine Aufgabe.

% \paragraph{Quantitative Methoden (Messen der Usability)}
% - System Usability Scale (SUS): Fragebogen zur schnellen, quantitativen Bewertung.

% \paragraph{Expertenbasierte Methoden}
% - Heuristische Evaluation: Experten prüfen das System anhand etablierter Heuristiken (z.B. von Nielsen).
