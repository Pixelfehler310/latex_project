% ### 2.1 Human-Centered Design (HCD) und Usability
\subsection{Human-Centered Design (HCD) und Usability}\label{subsec:hcd-usability}

% #### 2.1.1 Kognitive Aspekte der Interaktion (Psychologie, DOET und CLT)
\subsubsection{Kognitive Aspekte der Interaktion (Psychologie, DOET und CLT)}\label{subsubsec:kognitive-aspekte}

% * **Inhalt:**
% >
% > - **Problem**: Nutzersysteme mit schlechter Usability führen zu hoher kognitiver Last und Fehleranfälligkeit. -> Beispiel!
% > - **Norman's Konzepte**: Affordances, Signifiers, Mapping, Feedback, Constraints, Conceptual Model.
% > - **Cognitive Load Theory**: Extrinsische Last (durch schlechtes Tool) vs. Intrinsische Last (Aufgabe). Ziel: Reduktion der extrinsischen Last durch den visuellen Editor.
% > - **Gulf of Execution/Evaluation** (Norman): Die Kluft zwischen Nutzerziel und Systembedienung verringern.
% > - **Definitionen Usability & User Experience** (ISO 9241-11, ISO 9241-210 und weitere finden)
% > - Reihenfolge Variabel!

Softwaresysteme werden in der heutigen Welt immer komplexer. Um die Kunden zu überzeugen werden immer mehr Features und Funktionalitäten integriert. Dies führt jedoch oft dazu, dass die Bedienung dieser Systeme für die Nutzer zunehmend schwieriger und unübersichtlicher wird. % src: The Invisible Computer, Norman 1998, p.79
Diese Faktoren werden von der Digitalisierung und der zunehmenden Vernetzung und Automatisierung weiter verstärkt. % nosrc

Komplexer werdende Systeme sind eine Herausforderung für die Nutzer. Sie müssen sich mehr Abläufe, Funktionen, Bedienmöglichkeiten und Shortcuts merken. Das Problem dabei ist, dass der Mensch nur eine begrenzte Kapazität im Kurzzeitgedächtnis, oder auch Arbeitsgedächtnis genannt, hat, um Informationen zu verarbeiten und zu speichern. % verify in: Cognitive Load Theory, Sweller 1988
% src 2: DOET p. 100
1988 hat Sweller die Cognitive Load Theory (CLT) entwickelt, die sich mit der kognitiven Last beschäftigt, die bei der Informationsverarbeitung im Arbeitsgedächtnis entsteht. % src: Cognitive Load Theory, Sweller 1988
Die kognitive Last bezeichnet die Menge an mentaler Anstrengung, die erforderlich ist, um eine bestimmte Aufgabe zu bewältigen. % nosrc
Wenn die kognitive Last zu hoch wird und das Arbeitsgedächtnis des Nutzers überfordert ist, steigt die Wahrscheinlichkeit für Fehler bei der Bedienung des Systems. Nutzer übersehen wichtige Funktionen, vergessen Abläufe, verklicken sich, oder treffen falsche Entscheidungen. Dies führt zu Frustration, Unzufriedenheit und letztlich zur Ablehnung des Systems. % nosrc

Die CLT unterscheidet zwischen drei Arten von kognitiver Last: der intrinsischen Last, der extrinsischen Last und der germanen Last. % src: Cognitive Load Theory, Sweller 1988

Die intrinsische Last bezieht sich auf die Komplexität der zu erlernenden Aufgabe selbst. % src: Cognitive Load in UX Design 
% addsrc
Ist eine Aufgabe komplexer, erfordert sie mehr kognitive Ressourcen, um sie zu verstehen und zu bewältigen. % src: Cognitive Load in UX Design 

Die extrinsische Last hingegen ist die Last, die durch überflüssige, für das Systemverständnis irrelevante, Informationen entsteht, wie z.B. eine unübersichtliche Benutzeroberfläche, verwirrende Diagramme oder unnötige Ablenkungen. % src: Cognitive Load in UX Design 
% addsrc

Die germane Last bezieht sich auf die mentale Anstrengung, die der Nutzer aufwendet, um neues Wissen zu konstruieren und in das bestehende Wissen zu integrieren. % src Cognitive Load in UX Design
% addsrc

Die intrinsische Last ist oft unvermeidbar, da sie von der Natur der Aufgabe abhängt. Die germanische Last ist wünschenswert, da sie das Lernen und die Wissenskonstruktion fördert. Die extrinsische Last aber ist vermeidbar und kann durch eine gute Gestaltung des Systems reduziert werden. Ziel ist es, die extrinsische Last so gering wie möglich zu halten, um die kognitive Kapazität des Nutzers für die eigentliche Aufgabe freizusetzen und die Wahrscheinlichkeit von Fehlern zu minimieren. % src: Cognitive Load in UX Design
% addsrc
Da die extrinsische Last die einzige zu minimierende Last ist, ist in den folgenden Abschnitten die Reduzierung der kognitiven Last meist als Reduzierung der extrinsischen Last zu verstehen. % refine

Donald (Don) Norman, Professor für Kognitionswissenschaften und Informatik, hat in seinen Büchern The Design of Everyday Things und The Invisible Computer die wesentlichen psychologischen Vorgänge im menschlichen Gehirn bei der Interaktion von Objekten und Systemen analysiert und Ansätze vorgestellt die Nutzbarkeit der Systeme zu verbessern und die kognitive Last für die Nutzer zu reduzieren. 
Norman beschreibt 6 psychologische Konzepte, die bei der Gestaltung von benutzerfreundlichen Systemen berücksichtigt werden sollten.
Dazu gehören unter anderem Affordances, Signifiers, Mapping, Feedback, Constraints und das Conceptual Model. % src: The Design of Everyday Things, Norman 2013; The Invisible Computer, Norman 1998

Affordances sind Eigenschaften eines Systems, die Repräsentieren, was der User er tun kann. Sie definieren die Interaktionsangebote. % src: DOET p 27
Interaktionsangebote sind nicht immer offensichtlich. Eine Affordance bedeutet nicht automatisch, dass der Nutzer sie wahrnimmt. Bleibt die Interaktionsmöglichkeit unklar, kann sie nicht genutzt werden. Hier kommen Signifier ins Spiel. Signifier sind Hinweise oder Markierungen, die dem Nutzer die Interaktionsmöglichkeit, sowie deren Ausführung verdeutlichen. % src: DOET p 27-28
Sie sollen dem Nutzer kommunizieren, was der Zweck, der Aufbau und die Funktionsweise eines Objekts ist. % src: DOET p 30
Ein Signifier kann zum Beispiel ein Label auf einem Knopf sein, das suggeriert, was passiert, wenn dieser gedrückt wird, oder der Cursor der sich zum Pointer ändert, wenn er über einem Link hovert.
Eng verwandt mit den Signifiers sind Constraints. Anstatt zu zeigen was möglich ist, zeigen Constraints dem Nutzer was explizit nicht möglich ist.
Constraints begrenzen die möglichen Interaktionen, um Fehler zu vermeiden und die Bedienung zu leiten. Norman unterscheidet dabei zwischen physischen, kulturellen, semantischen und logischen Constraints. % src: DOET p p 128
Im Kontext von grafischen Benutzeroberflächen sind vor allem logische Constraints von Bedeutung. Sie nutzen das logische Verständnis des Nutzers über die Funktionsweise des Systems, um nur sinnvolle Aktionen zuzulassen. % src: DOET p 133
Ein klassisches Beispiel ist ein ausgegrauter (deaktivierter) Button in einem Formular, solange Pflichtfelder nicht ausgefüllt sind. Der Nutzer wird so aktiv daran gehindert, unvollständige Daten abzusenden, die zu Fehlern führen würden.

Das vierte Konzept ist Mapping. Mapping beschreibt die Beziehung zwischen den Steuerelementen eines Systems und den Aktionen, die sie auslösen. % src: DOET p 37
Gutes Mapping bedeutet, dass die Anordnung der Steuerelemente intuitiv und logisch ist, sodass der User leicht verstehen kann, welche Steuerung welche Aktion auslöst. % src: DOET p 38

% todo Beispiel für Mapping

Feedback ist ein weiteres wichtiges Konzept. Es beschreibt die Rückmeldung, die ein System dem Nutzer gibt, nachdem eine Aktion ausgeführt wurde. % src: DOET p 40
Feedback informiert den Nutzer über den aktuellen Zustand des Systems oder die Ergebnisse seiner Aktionen. % src: DOET p 40

Feedback ist entscheidend, damit der User versteht, was im System passiert, und um ihm Sicherheit bei der Interaktion zu geben. Dauert Feedback zu lange oder fehlt es ganz, kann der User verwirrt werden und Fehler machen. % src: DOET p 41
Außerdem muss Feedback geplant und priorisiert werden, da auch zu viel oder unnötiges Feedback den User überfordern, ablenken und verwirren kann. % src: DOET p 41 
Dies kann in verschiedenen Formen geschehen, wie visuelle Hinweise, akustische Signale oder haptische Rückmeldungen. % nosrc

Das sechste Konzept ist das Conceptual Model. Ein Conceptual Model ist eine mentale Repräsentation, die der User von einem System entwickelt, basierend auf seinen Erfahrungen, seinem Wissen und den Informationen, die das System bereitstellt, über die Funktionsweise des Systems % src: DOET p 42   
% todo Satz verbessern
Oft entstehen diese Modelle durch das Lernen des Systems durch Interaktion und Ausprobieren, durch den Informationstransfer von Person zu Person, oder durch das Lesen von Dokumentationen oder ähnlichen Systembeschreibungen. % src: DOET p 43
Das Conceptual Model ist eng verbunden mit der germanen Last der CLT, da es das gelernte Wissen des Nutzers über das System repräsentiert und ihm hilft, neue Informationen zu integrieren. % refine

Diese Prinzipien erklären die Wahrnehmung eines Nutzers bei der Interaktion mit einem System. Wenn man diese Verstanden hat kann man sie gezielt einsetzen, um die Wahrnehmung des Nutzers zu lenken, und dadurch die Bedienung zu vereinfachen und die kognitive Last zu reduzieren. % refine

Nach Hutchins, Hollan und Norman gibt es zwei „Gulfs“, die bei der Interaktion zwischen Mensch und System überwunden werden müssen: den Gulf of Execution und den Gulf of Evaluation. % search for: User Centered System Design, 1986
% src: Direct Manipulation Interfaces, Hutchins, Hollan, Norman 1985
% addsrc


% todo Beispiele für die psychologischen Konzepte einfügen (Affordances, Signifiers, Mapping, Feedback, Constraints, Conceptual Model) (jedes einzeln, direkt beim inhaltlichen Abschnitt)

% todo Feed-Forward


% #### 2.1.2 Human Centered Design (HCD) Prinzipien
\subsubsection{Human Centered Design (HCD) Prinzipien}\label{subsubsec:hcd-prinzipien}

% * **Inhalt:**
%  > Der HCD Prozess mit seinen vier Phasen (Verstehen und Spezifizieren des Nutzungskontextes, Spezifizieren der Nutzungsanforderungen, Gestalten der Lösung, Evaluieren der Lösung) wird vorgestellt. Der Fokus liegt dabei auf den Prinzipien des nutzerzentrierten Designs, die in jeder Phase angewendet werden sollten, um sicherzustellen, dass die Bedürfnisse und Anforderungen der Nutzer im Mittelpunkt stehen.
% > Nach Iso Norm und ggf weitere quellen (Norman, Shneiderman)

% #### 2.1.3 Gestaltungsprinzipien für interaktive Systeme
\subsubsection{Gestaltungsprinzipien für interaktive Systeme}\label{subsubsec:gestaltungsprinzipien}

% * **Inhalt:**
% > - **Aus 2.1.1 ableitbare Designprinzipien**
% > - **kognitive zusammenfassung der wichtigsten Heuristiken, mit Fokus auf ISO NORM 9241:210:2010**: (Additiv und als BOnus quellen: Norman, Shneiderman, Nielsen, Honeycomb)
% > - **Feedback & Feed-Forward**: Nutzerführung.
% > - **Konsistenz**: Erwartungskonformität.
% >   Dialoggestaltungs Prinzipien (9241-110)
% > - explizit NICHT hier drin: Direkte und indirekte Manipulation und WYSIWYG (kommt in 2.3.1)

% #### 2.1.4 Methoden der Usability-Evaluation
\subsubsection{Methoden der Usability-Evaluation}\label{subsubsec:usability-evaluation}

% * **Inhalt:**
% > - **UX Honeycomb** (Peter Morville): Useful, Usable, Desirable. (WICHTIG für die spätere Nutzwertanalyse, liefert die Kategorien)
% > - **Heuristische Evaluation** (Nielsen).
% > - **Usability Testing** (Nutzer-basierte Evaluation).
% > - **Qualitative vs. Quantitative Methoden**.
% > - **Ggf weitere Methoden wenn sinnvoll**
