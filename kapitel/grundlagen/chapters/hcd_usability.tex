% 2.3 Human-Centered Design (HCD) und Usability
\subsection{Human-Centered Design (HCD) und Usability}\label{subsec:hcd-usability}

% TODO: Ja, das kann grundsätzlich HCD sein – wenn bestimmte Bedingungen eingehalten werden. Aber: Dein Beispiel ist eher „lightweight HCD“ oder „User-informed Design“, nicht der vollständige, formal definierte HCD-Prozess nach ISO 9241-210.
% TODO also noch User Informed Design ergänzen, / den unterschied erklären!!!
% ! ❗ Potenzielle Lücken: Ist beim Designer wirklich ein Tiefes Nutzungsverständnis vorhanden? ; ist die repräsentative Gruppe wirklich nah an den Endnutzern? ; Wo Evaluation gegen Nutzungskontext???



% Einleitung:
% - Definition HCD (ISO 9241-210): Prozess, der den Menschen in den Mittelpunkt stellt.
% - Ziel: Gebrauchstaugliche (usable) und nützliche (useful) Systeme schaffen.
% - Abgrenzung zu Usability: HCD ist der Prozess, Usability ein Qualitätsmerkmal.

% TODO Einleitungssatz schreiben // Human Centered Design, User Experience und Usability sind bEgriffe die die Softwareentwicklung und das Interaktionsdesign schon seit Jahren prägen. ...

Human-Centered Design ist ein Ansatz zur Gestaltung interaktiver Systeme, bei dem der Mensch in den Mittelpunkt gestellt wird. Das Ziel ist es, die Bedienbarkeit der Systeme zu schon im Entwicklungsprozess zu verbessern, in dem Wissen über die Nutzer, ihre Aufgaben und die Umgebung systematisch analysiert und einbezogen wird. (ISO 9241-210, p2,2.7). Das Ergebnis eines auf HCD basierenden Entwicklungsprozesses sind bestenfalls Systeme, die eine hohe Usability aufweisen, also Systeme, die effektiv, effizient und zufriedenstellend von den Nutzern verwendet werden können (ISO 9241-210, p3, 2.13). 

Gute Usability und die Nutzung von HCD-Prinzipien können helfen die Akzeptanz und Zufriedenheit der Nutzer zu erhöhen, die Produktivität zu steigern und Fehler zu reduzieren. Außerdem können sie die Trainigs und Supportkosten senken, da Systeme intuitiver und leichter zu erlernen sind. Weniger Frustration bei der Nutzung durch gutes Design führt zudem zu einer positiveren Wahrnehmung des Produkts und der Marke.

Die ISO Norm 9241-210 beschreibt sechs Grundprinzipien des Human-Centered-Design-Prozesses. 
1. Der Entwurf basiert auf einem klaren Verständnis der Nutzer, deren Aufgaben und der Nutzungsumgebung.
2. Die Nutzer werden aktiv in den Entwurfs- und Entwicklungsprozess einbezogen.
3. Der Entwurf wird iterativ entwickelt.
4. Alle Aspekte der User Experience werden berücksichtigt.
5. Das Design wird von einem Team erstellt, das diversifizierte Fähigkeiten und Perspektiven mitbringt.  % TODO den Satz kann man verbessern % TODO ggf ausführen welche Fähigkeiten es braucht

Ein beispielhafter HCD-Prozess könnte in etwa so aussehen: % TODO Beispiel ausfeilen
Zunächst muss man analysieren und verstehen, wer die Nutzer sind, welche Aufgaben sie erledigen müssen und in welchem Kontext sie das System verwenden. Basierend auf diesem Verständnis werden Anforderungen definiert, die das Design leiten.
Anschließend werden erste Entwürfe entwickelt, die diese Anforderungen erfüllen. Der frühe Entwurf wird dann bereits an die User kommuniziert um Feedback der Nutzer einzuholen. Dieser Schritt nennt sich Validierung des Designs. Die Nutzer bereits früh in den Prozess einzubinden ermöglicht es, aufkommende Probleme frühzeitig erkennen und beheben zu können.
Mithilfe des User-Feedbacks wird das Produkt dann angepasst und verbessert. 

Diese Schritte werden mehrfach durchlaufen und so wird das System kontinuierlich besser. Aufgrund dieser iterativen Struktur sind Agile Entwicklungsmethoden gut geeignet, um HCD-Prinzipien umzusetzen.

% TODO ROI erklären, also warum sich HCD lohnt

Die Evaluation der Usability und der User Experience ist ein wichtiger Bestandteil des HCD-Prozesses. Doch um bewerten zu können, ob ein System eine gute User Experience oder Usability aufweist, muss man zunächst verstehen, was diese Begriffe bedeuten und wie sie gemessen werden können.

\subsubsection{Gestaltungsprinzipien für interaktive Systeme}\label{subsubsec:gestaltungsprinzipien}
Don Norman beschreibt in seinem Buch "The Design of Everyday Things" sechs psychologische Konzepte, deren Verständnis für gutes Design unerlässlich ist, da sie erklären, wie Nutzer mit Objekten interagieren und wie sie diese wahrnehmen. % TODO Anführungszeichen prüfen

% ? Vllt so ein Satz: Im Originalwerk beziehen sich die Konzepte auf Alltagsgegenstände. Sie gelten aber ebenso gut für interaktive Systeme und Software, daher werden sie hier auf Systeme bezogen


% TODO Nutzer == User, einmal einheitlich machen am besten (ich mag ja user lieber)

Das erste Konzept beschreibt Affordances. Affordances sind Eigenschaften eines Systems, die Repräsentieren, was der User er tun kann, also Interaktionsangebote. % ~ DOET p 27

Interaktionsangebote sind nicht immer offensichtlich. Wenn ein Objekt eine Affordance besitzt, bedeutet das nicht automatisch, dass der User diese auch wahrnimmt. Wenn dem User die Interaktionsmöglichkeit nicht klar ist, kann er sie auch nicht nutzen. Daher gibt es Signifiers. 

Signifiers sind Hinweise oder Markierungen, die dem User zeigen, dass und wie er mit einem Objekt interagieren kann. % TODO Satz verbessern

% TODO Contraints, s. DOET Chapter 3 und 4

Das zweite Konzept sind Signifiers, also Hinweise oder Markierungen, die dem User zeigen, wo und wie er mit einem Objekt interagieren kann.% ~ DOET p 28 
Sie sollen dem Nutzer kommunizieren, was der Zweck, der Aufbau und die Funktionsweise eines Objekts ist. % ~ DOET p 30
Ein Beispiel hierfür ist ein Pfeil auf einem Knopf, der anzeigt, dass dieser gedrückt werden kann, oder der Cursor der sich ändert, wenn er über einem Link hovert.

Das dritte Konzept ist Mapping. Mapping beschreibt die Beziehung zwischen den Steuerelementen eines Systems und den Aktionen, die sie auslösen. % ~ DOET p 37
Gutes Mapping bedeutet, dass die Anordnung der Steuerelemente intuitiv und logisch ist, sodass der User leicht verstehen kann, welche Steuerung welche Aktion auslöst. % ~ DOET p 38

Das vierte Konzept ist Feedback. Feedback ist die unmittelbare Rückmeldung % ~ DOET p 40
, die ein System dem User gibt, um ihn über den aktuellen Zustand oder die Ergebnisse seiner Aktionen zu informieren. % ~ DOET p 40

Feedback ist entscheidend, damit der User versteht, was im System passiert, und um ihm Sicherheit bei der Interaktion zu geben. Dauert Feedback zu lange oder fehlt es ganz, kann der User verwirrt werden und Fehler machen. % ~ DOET p 41

Außerdem muss Feedback geplant und priorisiert werden, da auch zu viel oder unnötiges Feedback den User überfordern, ablenken und verwirren kann. % ~ DOET p 41 

Das fünfte Konzept ist das Conceptual Model. Ein Conceptual Model ist eine mentale Repräsentation, die der User von einem System entwickelt, basierend auf seinen Erfahrungen, seinem Wissen und den Informationen, die das System bereitstellt, über die Funktionsweise des Systems % ~ DOET p 42 % TODO Satz verbessern
Oft entstehen diese Modelle durch das Lernen des Systems durch Interaktion und Ausprobieren, durch den Informationstransfer von Person zu Person, oder durch das Lesen von Dokumentationen oder ähnlichen Systembeschreibungen. % ~ DOET p 43


% TODO Software Beispiele für alle fünf Prinzipien am besten in einem Software System


% Hier die wichtigsten Prinzipien als Grundlage für gutes Interaktionsdesign vorstellen.

Aus den oben genannten Konzepten lassen sich grundlegende Designprinzipien ableiten, die als Leitlinien für die Gestaltung interaktiver Systeme dienen können. In der ISO Norm 9241-110 werden sieben Dialogprinzipien beschrieben, die eng mit den oben genannten Konzepten verwandt sind und die Gestaltung benutzerfreundlicher Systeme unterstützen: 

Aufgabenangemessenheit: Systeme sollen AnwenderInnen dabei unterstützen, Aufgaben einfach und direkt erledigen zu können, ohne komplizierte Bedienung. Beispiel: Kontextsensitive Dialoge als Hilfestellung während eines Online-Bestellvorgangs.
Selbstbeschreibungsfähigkeit: Für NutzerInnen soll zu jedem Zeitpunkt klar sein, in welchem und an welcher Stelle eines Dialogs er oder sie sich befinden und welche Handlungen wie ausgeführt werden können. Beispiel: Werden Eingaben vom System erwartet, soll klar sein, wie diese bereit gestellt werden sollen (z.B. TT.MM.JJJJ für eine Eingabe des Datums).
Erwartungskonformität: BenutzerInnen sollen aufgrund ihrer Vorerfahrungen und allgemein üblichen Konventionen behandelt werden. Beispiel: Berücksichtigung von kulturellen und sprachlichen Bestimmungen.
Lernförderlichkeit: Unterstützen Dialoge beim Erlenen der Nutzung von Systemen, gelten diese als lernförderlich. Beispiel: Anbieten von geeigneter Unterstützung wie Hilfestellungen durch Drücken einer Taste.
Steuerbarkeit: Kann ein Nutzer den Dialogablauf starten und Richtung sowie Geschwindigkeit ändern, bis ein Ziel erreicht ist, gilt der Dialog als steuerbar. Beispiel: Möglichkeit der Verwendung jedes verfügbaren Ein-/Ausgabemittels, wo es geeignet ist.
Fehlertoleranz: Fehlertolerante Dialoge liegen vor, wenn fehlerhafte Eingaben nur mit geringfügigem Korrekturaufwand durch NutzerInnen beseitigt werden können. Beispiel: Können Systeme automatisiert Fehler korrigieren, sollten sie die AnwenderInnen über die Ausführung informieren.
Individualisierbarkeit: Können Personen Dialoge nach ihren Fähigkeiten und Bedürfnissen anpassen und die Darstellung von Informationen ändern, gelten diese als individualisierbar. Beispiel: NutzerInnen können zwischen unterschiedlichen Darstellungsformen wählen, wenn es für sie zweckmäßig ist (z.B. größere Schriftzeichen bei Sehbehinderung). (Vgl. DIN EN ISO 9241-110, 2008, S. 8ff.)



% \paragraph{Grundlegende Designprinzipien}
% - Sichtbarkeit (Visibility): Was kann ich tun?
% - Rückmeldung (Feedback): Was ist passiert?
% - Konsistenz (Consistency): Gleiche Dinge sollten gleich aussehen und funktionieren.
% - Affordances & Signifiers: Was ein Objekt kann (Affordance) und wie das kommuniziert wird (Signifier).

% \paragraph{Konzepte für Editoren und visuelle Werkzeuge}
% - Direkte Manipulation (Shneiderman): Gefühl der direkten Kontrolle über Objekte.
% - WYSIWYG (What You See Is What You Get): Die Darstellung im Editor entspricht dem Endergebnis.

% \paragraph{Heuristiken als praktische Leitlinien}
% - Nielsen's 10 Usability Heuristics als anerkannter Standard.
% - Shneiderman's "Golden Rules of Interface Design".

\subsubsection{Methoden der Usability-Evaluation}\label{subsubsec:usability-methoden}

% Überblick über Methoden, um die Usability eines Systems zu bewerten.
% Unterscheidung: Formative (während der Entwicklung) vs. summative (am Ende) Evaluation.

% \paragraph{Qualitative Methoden (Verstehen des "Warum")}
% - Thinking Aloud: Nutzer verbalisieren ihre Gedanken bei der Nutzung.
% - Cognitive Walkthrough: Experten simulieren den Weg eines Nutzers durch eine Aufgabe.

% \paragraph{Quantitative Methoden (Messen der Usability)}
% - System Usability Scale (SUS): Fragebogen zur schnellen, quantitativen Bewertung.

% \paragraph{Expertenbasierte Methoden}
% - Heuristische Evaluation: Experten prüfen das System anhand etablierter Heuristiken (z.B. von Nielsen).
