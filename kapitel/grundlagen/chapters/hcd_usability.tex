% ### 2.1 Human-Centered Design (HCD) und Usability
\subsection{Human-Centered Design (HCD) und Usability}\label{subsec:hcd-usability}

% #### 2.1.1 Kognitive Aspekte der Interaktion (Psychologie, DOET und CLT)
\subsubsection{Kognitive Aspekte der Interaktion (Psychologie, DOET und CLT)}\label{subsubsec:kognitive-aspekte}

% * **Inhalt:**
% >
% > - **Problem**: Nutzersysteme mit schlechter Usability führen zu hoher kognitiver Last und Fehleranfälligkeit. -> Beispiel!
% > - **Norman's Konzepte**: Affordanzen, Signifiers, Mapping, Feedback, Constraints, Conceptual Model.
% > - **Cognitive Load Theory**: Extrinsische Last (durch schlechtes Tool) vs. Intrinsische Last (Aufgabe). Ziel: Reduktion der extrinsischen Last durch den visuellen Editor.
% > - **Gulf of Execution/Evaluation** (Norman): Die Kluft zwischen Nutzerziel und Systembedienung verringern.
% > - **Definitionen Usability & User Experience** (ISO 9241-11, ISO 9241-210 und weitere finden)
% > - Reihenfolge Variabel!

% #### 2.1.2 Human Centered Design (HCD) Prinzipien
\subsubsection{Human Centered Design (HCD) Prinzipien}\label{subsubsec:hcd-prinzipien}

% * **Inhalt:**
% > - **Aus 2.1.1 ableitbare Designprinzipien**
% > - **kognitive zusammenfassung der wichtigsten Heuristiken, mit Fokus auf ISO NORM 9241:210:2010**: (Additiv und als BOnus quellen: Norman, Shneiderman, Nielsen, Honeycomb)
% > - **Feedback & Feedforward**: Nutzerführung.
% > - **Konsistenz**: Erwartungskonformität.
% >   Dialoggestaltungs prinzipien (9241-110)
% > - explizit NICHT hier drin: Direkte und indirekte Manipulation und WYSIWYG (kommt in 2.3.1)


% #### 2.1.3 Gestaltungsprinzipien für interaktive Systeme
\subsubsection{Gestaltungsprinzipien für interaktive Systeme}\label{subsubsec:gestaltungsprinzipien}

% * **Inhalt:**
% > - **Aus 2.1.1 ableitbare Designprinzipien**
% > - **kognitive zusammenfassung der wichtigsten Heuristiken, mit Fokus auf ISO NORM 9241:210:2010**: (Additiv und als BOnus quellen: Norman, Shneiderman, Nielsen, Honeycomb)
% > - **Feedback & Feedforward**: Nutzerführung.
% > - **Konsistenz**: Erwartungskonformität.
% >   Dialoggestaltungs prinzipien (9241-110)
% > - explizit NICHT hier drin: Direkte und indirekte Manipulation und WYSIWYG (kommt in 2.3.1)

% #### 2.1.4 Methoden der Usability-Evaluation
\subsubsection{Methoden der Usability-Evaluation}\label{subsubsec:usability-evaluation}

% * **Inhalt:**
% > - **UX Honeycomb** (Peter Morville): Useful, Usable, Desirable. (WICHTIG für die spätere Nutzwertanalyse, liefert die Kategorien)
% > - **Heuristische Evaluation** (Nielsen).
% > - **Usability Testing** (Nutzerbasierte Evaluation).
% > - **Qualitative vs. Quantitative Methoden**.
% > - **Ggf weitere Methoden wenn sinnvoll**
