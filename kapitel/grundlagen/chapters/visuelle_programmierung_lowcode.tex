% 2.2 Visuelle Programmierung und Low-Code-Plattformen
\subsection{Visuelle Programmierung und Low-Code-Plattformen}\label{subsec:vp-lc}

% Einordnung und Verbindung
Visuelle Editoren dienen als Lösungsansatz zur Erhöhung der Abstraktion und zur Vereinfachung der Anwendungsentwicklung. In modernen Entwicklungsprozessen schlagen Low-Code/No-Code-Ansätze eine Brücke zwischen fachlicher Modellierung und technischer Umsetzung.

\subsubsection{Konzepte der Visuellen Programmierung}\label{subsubsec:vp-konzepte}
\textbf{Inhalte (Überblick):}
\begin{itemize}
  \item Grundkonzepte: Knoten/Blöcke, Verbindungen, Daten- vs. Kontrollfluss
  \item Direkte Manipulation von Artefakten, WYSIWYG als Zielbild
  \item Mapping von visuellen Konstrukten auf ausführbare Artefakte (z.\,B. Code, Konfiguration, Modelle)
\end{itemize}

\paragraph{Schlagworte für Literaturrecherche}
% TODO: eigene Schlagworte ergänzen
\texttt{Visual Programming}, \texttt{Dataflow vs. Control Flow}, \texttt{Node-based Editors}, \texttt{WYSIWYG Editors}

\paragraph{Quellen und Notizen}
\begin{itemize}
  \item TODO: Referenzen zu visuellen Programmiersystemen, node-based Editor-Konzepten.
\end{itemize}

\subsubsection{Low-Code Plattformen}\label{subsubsec:lowcode}
\textbf{Inhalte (Überblick):}
\begin{itemize}
  \item Definition von Low-Code als Synthese aus MDSE-Prinzipien und visueller Entwicklung
  \item Architekturen bekannter NCLC-Plattformen: OutSystems, Mendix, Microsoft Power Apps
  \item Rolle des \glqq Citizen Developers\grqq{}
  \item Abgrenzung No-Code (NC), Low-Code (LC), Pro-Code (PC)
\end{itemize}

\paragraph{Definition und Abgrenzung}
Low-Code Plattformen kombinieren visuelle Modellierung, vorgefertigte Komponenten und optionalen Code, um die Entwicklung zu beschleunigen. No-Code zielt auf vollständige Entwicklung ohne Programmierung ab, während Pro-Code klassische, vollumfängliche Programmierung bezeichnet. LC kann bei Bedarf um Pro-Code erweitert werden (Hybrid).

\begin{center}
\begin{tabular}{lccc}
\textbf{Merkmal} & \textbf{NC} & \textbf{LC} & \textbf{PC} \\
\hline
Grundwissen & Kaum / Gar keins & Gering / Moderat & Vollständig \\
Funktionale Abdeckung & bis zu 95\% & 90--95\% + 100\% mit User-Skripten & 100\% \\
\end{tabular}
\end{center}

\begin{quote}
``Low-Code und No-Code sind Methoden zum Konzipieren und Entwickeln von Apps mit intuitiven Drag-and-Drop-Tools. Weil das auch ohne Programmierkenntnisse möglich ist, sinkt der Bedarf an Fachkräften für die Programmierung erheblich.'' \textit{(Quelle nachtragen)}
\end{quote}

\begin{quote}
``Mit Low Code kann man nicht 100\% der Software abbilden, sondern meist nur 90--98\%.'' \textit{(Quelle nachtragen)}
\end{quote}

\paragraph{Schlagworte für Literaturrecherche}
\texttt{Low-Code Application Platform (LCAP)}, \texttt{Citizen Development}, \texttt{Visual Development Environment}, \texttt{Rapid Application Development (RAD)}

\paragraph{Quellen und Notizen}
\begin{itemize}
  \item TODO: Produktdokumentationen (OutSystems, Mendix, Microsoft Power Apps); Marktanalysen (z.\,B. Gartner, Forrester).
\end{itemize}

\subsubsection{Analyse und Einordnung vorhandener visueller Editoren und LCDPs}\label{subsubsec:lcdp-analyse}
\textbf{Inhalte (Überblick):}
\begin{itemize}
  \item Kriterienkatalog (Modellierungstiefe, Erweiterbarkeit, Integration, Deployment, Governance)
  \item Vergleich/Einordnung: OutSystems, Mendix, Microsoft Power Apps (Stärken/Schwächen)
  \item Rolle und Grenzen des Citizen Developments; LC/PC-Hybride
\end{itemize}

\paragraph{Schlagworte für Literaturrecherche}
\texttt{Platform Comparison}, \texttt{Extensibility}, \texttt{Integration}, \texttt{Governance}, \texttt{DevOps in Low-Code}

\paragraph{Quellen und Notizen}
\begin{itemize}
  \item TODO: Vergleichsstudien, Whitepaper, Fallstudien; Produkt- und Community-Ressourcen.
\end{itemize}
