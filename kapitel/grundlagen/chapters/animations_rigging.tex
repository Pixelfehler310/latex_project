\subsection{Animation-Rigging} % 500 T
Mit dem Animation-Rigging-Paket von Unity können prozedurale Animationen erstellt und bearbeitet werden, die sich dynamisch zur Laufzeit oder während einer Animation verändern lassen. Dadurch sind flexible Bewegungen innerhalb einer Animation sowie interaktive Animationen mit anderen Entitäten möglich. Dadurch werden die Funktionen des von Anfang an eingebauten Animation-Systems erweitert und verbessert. Das Paket wird von Unity selbst entwickelt und ist aktuell in der Version 1.3.0 verfügbar. Es lässt sich wie andere Pakete über den Unity Package Manager installieren. \footcite[\vglf][]{unity.animationRigging}
Interaktive Animationen sind in vielen Spielen ein wichtiger Bestandteil für ein immersives und dynamisches Spielerlebnis. Sie ermöglichen es beispielsweise, dass Spielcharaktere realistisch mit der Umgebung interagieren, indem sie Objekte aus verschiedenen Winkeln aufheben, ohne dass exakt vordefinierte Animationen verwendet werden müssen. In dem Videospiel Half-Life 2 wird dies zum Beispiel genutzt, um den Kopf eines \ac{NPC}, immer in Richtung des Spielers zu drehen und so eine realistischere Interaktion zu ermöglichen. In der \autoref{fig:hl2_animation_rigging} ist ein Beispiel für eine solche Animation zu sehen.

\begin{figure}[H]
  \caption[Dynamische Drehung des Kopfes]{Dynamische Drehung des Kopfes}\label{fig:hl2_animation_rigging}
  \includegraphics[width=1.0\textwidth]{hl2_animation_rigging.png}
\end{figure}

Als Rigging bezeichnet man das Erstellen einer Struktur, die für die Animation von Charakteren oder Objekten verwendet wird. Diese Struktur besteht aus mehreren miteinander verbundenen Knochenelementen, die ein Skelett, auch Rig genannt, bilden. Mithilfe des Skeletts können anschließend die einzelnen Teile des Charakters oder Objekts animiert werden. Die Transformationen der Knochenelemente beeinflussen das Mesh des Modells, sodass sich das gesamte Objekt entsprechend der Animation bewegt. \footcite[\vglf][]{riggingBasics} \footcite[\vglf][]{characterRiggingForGames}
Das Unity-Animation-Rigging-Paket umfasst neben dem Rigging selbst auch constraint-basierte Steuerungen und prozedurale Anpassungen. Dies ermöglicht einerseits das gezielte Beeinflussen bestimmter Teile des Skeletts, zum Beispiel, um Objekte vom Boden aufzuheben oder den Kopf nach einem Objekt auszurichten, und andererseits die Übersteuerung von Animationen, um Überblendungen zwischen Animationen zu ermöglichen oder Animationen dynamisch an bestimmte Situationen anzupassen. \footcite[\vglf][]{unity.animationRigging}
Constraints sind Regeln, die definieren, wie sich die Teile beziehungsweise die Knochenelemente verhalten sollen. Der Multi-Aim-Constraint richtet beispielsweise mehrere Knochenelemente auf ein ausgewähltes Ziel aus. Ein Rig kann aus mehreren Constraints bestehen und besitzt somit eine Sammlung von Constraints beziehungsweise Regeln. Die Rig-Layer erlauben es, verschiedene Rigs mit Priorisierung zu kombinieren. Für die einzelnen Rigs und Constraints eines Rigs können Gewichtungen, auch Weights genannt, festgelegt werden, um anzugeben, wie stark die einzelnen Constraints auf das Rig und das Rig auf das Mesh wirken sollen.
In Unity können Rigs mithilfe der Rig-Builder-Komponente erstellt werden. Diese verwaltet alle Rigs eines GameObjects beziehungsweise 3D-Modells und ermöglicht es, diese zu kombinieren und zu priorisieren. Die Rigs selbst werden über die Rig-Komponente erstellt und anschließend in der Rig-Builder-Komponente hinzugefügt. Ein Rig besitzt eine Kopie aller Knochenelemente des 3D-Modells sowie eine Eingabe zur Angabe der Gewichtung des gesamten Rigs. Die einzelnen Constraints werden als weitere Komponenten dem Rig hinzugefügt. Dafür gibt es eine Auswahl an verschiedenen Komponenten, die jeweils für einen bestimmten Constraint-Typ zuständig sind. Zudem muss für jeden Constraint mindestens ein Knochenelement angegeben werden, das beeinflusst werden soll. Wie das Rig besitzt jeder Constraint ebenfalls eine Eingabe für die Gewichtung des Constraints. Diese gibt an, wie stark der Constraint auf das Rig und damit auf das Mesh wirkt. \footcite[\vglf][]{unity.animationRiggingManual}