%-----------------------------------
% Define document and include general packages
%-----------------------------------
% Tabellen- und Abbildungsverzeichnis stehen normalerweise nicht im
% Inhaltsverzeichnis. Gleiches gilt für das Abkürzungsverzeichnis (siehe unten).
% Manche Dozenten bemängeln das. Die Optionen 'listof=totoc,bibliography=totoc'
% geben das Tabellen- und Abbildungsverzeichnis im Inhaltsverzeichnis (toc=Table
% of Content) aus.
% Da es aber verschiedene Regelungen je nach Dozent geben kann, werden hier
% beide Varianten dargestellt.
%\documentclass[12pt,oneside,titlepage,listof=totoc,bibliography=totoc]{scrartcl}
\documentclass[12pt,oneside,titlepage]{scrartcl}

%-----------------------------------
% Dokumentensprache
%-----------------------------------
%\def\FOMEN{}% Auskommentieren um die Dokumentensprache auf englisch zu ändern
\newif\ifde
\newif\ifen

%-----------------------------------
% Meta informationen
%-----------------------------------
%-----------------------------------
% Meta Informationen zur Arbeit
%-----------------------------------

% Autor
\newcommand{\myAutor}{Timo Kappel, Simon Sucker, Mario Peter Hauf}

% Adresse
\newcommand{\myAdresse}{Holtkämperheide 11 45329 Essen, Beckstadtstraße 42 45472 Mülheim an der Ruhr , Kanarienberg 11 45329 Essen}

% Titel der Arbeit
\newcommand{\myTitel}{Spielentwicklung in Unity}

% Betreuer
\newcommand{\myBetreuer}{Prof. Dr. Adem Alparslan}

% Lehrveranstaltung
\newcommand{\myLehrveranstaltung}{Praxisprojekt III}

% Matrikelnummer
\newcommand{\myMatrikelNrT}{668477}

\newcommand{\myMatrikelNrS}{669007}

\newcommand{\myMatrikelNrM}{640667}
% Ort
\newcommand{\myOrt}{Essen}

% Datum der Abgabe
\newcommand{\myAbgabeDatum}{\today}

% Semesterzahl
\newcommand{\mySemesterZahl}{6}

% Name der Hochschule
\newcommand{\myHochschulName}{FOM Hochschule für Oekonomie \& Management}

% Standort der Hochschule
\newcommand{\myHochschulStandort}{Essen}

% Studiengang
\newcommand{\myStudiengang}{Informatik}

% Art der Arbeit
\newcommand{\myThesisArt}{Projektarbeit}

% Zu erlangender akademische Grad
\newcommand{\myAkademischerGrad}{Bachelor of Science (B.Sc.)}

% Firma
\newcommand{\myFirma}{}


\ifdefined\FOMEN
%Englisch
\entrue
\usepackage[english]{babel}
\else
%Deutsch
\detrue
\usepackage[ngerman]{babel}
\fi


\newcommand{\langde}[1]{%
   \ifde\selectlanguage{ngerman}#1\fi}
\newcommand{\langen}[1]{%
   \ifen\selectlanguage{english}#1\fi}
\usepackage[utf8]{luainputenc}
\langde{\usepackage[babel,german=quotes]{csquotes}}
\langen{\usepackage[babel,english=british]{csquotes}}
\usepackage[T1]{fontenc}
\usepackage{fancyhdr}
\usepackage{fancybox}
\usepackage[a4paper, left=4cm, right=2cm, top=4cm, bottom=2cm]{geometry}
\usepackage{graphicx}
\usepackage{colortbl}
\usepackage{caption}
\captionsetup[table]{
    format=plain,
    labelfont=bf,
	font=small,
    justification=centering,
    labelsep=colon
}
\captionsetup[figure]{
    format=plain,
    labelfont=sf,
	font=small, 
    justification=centering,
    labelsep=colon
}
%\usepackage[capposition=top]{floatrow}
\usepackage[capposition=bottom]{floatrow}
\floatsetup[table]{capposition=top}
\usepackage{array}
\usepackage{float}      %Positionierung von Abb. und Tabellen mit [H] erzwingen
\usepackage{footnote}
% Darstellung der Beschriftung von Tabellen und Abbildungen (Leitfaden S. 44)
% singlelinecheck=false: macht die Caption linksbündig (statt zentriert)
% labelfont auf fett: (Tabelle x.y:, Abbildung: x.y)
% font auf fett: eigentliche Bezeichnung der Abbildung oder Tabelle
% Fettschrift laut Leitfaden 2018 S. 45
\usepackage[singlelinecheck=false, labelfont=bf, font=bf]{caption}
\usepackage{caption}
\usepackage{enumitem}
\usepackage{amssymb}
\usepackage{mathptmx}
%\usepackage{minted} %Kann für schöneres Syntax Highlighting genutzt werden. ACHTUNG: Python muss installiert sein.
\usepackage[scaled=0.9]{helvet} % Behebt, zusammen mit Package courier, pixelige Überschriften. Ist, zusammen mit mathptx, dem times-Package vorzuziehen. Details: https://latex-kurs.de/fragen/schriftarten/Times_New_Roman.html
\usepackage{courier}
\usepackage{amsmath}
\usepackage[table]{xcolor}
\usepackage{marvosym}			% Verwendung von Symbolen, z.B. perfektes Eurozeichen

\renewcommand\familydefault{\sfdefault}
\usepackage{ragged2e}

% Mehrere Fussnoten nacheinander mit Komma separiert
\usepackage[hang,multiple]{footmisc}
\setlength{\footnotemargin}{1em}

% todo Aufgaben als Kommentare verfassen für verschiedene Editoren
\usepackage{todonotes}

% Verhindert, dass nur eine Zeile auf der nächsten Seite steht
\setlength{\marginparwidth}{2cm}
\usepackage[all]{nowidow}




%-----------------------------------
% Farbdefinitionen
%-----------------------------------
\definecolor{darkblack}{rgb}{0,0,0}
\definecolor{dunkelgrau}{rgb}{0.8,0.8,0.8}
\definecolor{hellgrau}{rgb}{0.0,0.7,0.99}
\definecolor{mauve}{rgb}{0.58,0,0.82}
\definecolor{dkgreen}{rgb}{0,0.6,0}
\definecolor{marineblue}{rgb}{0.51, 0.76, 0.8}
\definecolor{lightblue}{rgb}{0.4, 0.52, 0.8}
\definecolor{darkyellow}{rgb}{1, 0.8, 0.2}
\definecolor{darkorange}{rgb}{1, 0.5, 0.0}

%-----------------------------------
% Pakete für Tabellen
%-----------------------------------
\usepackage{epstopdf}
\usepackage{nicefrac} % Brüche
\usepackage{multirow}
\usepackage{rotating} % vertikal schreiben
\usepackage{mdwlist}
\usepackage{tabularx}% für Breitenangabe
\usepackage{algorithm}
\usepackage{algorithmic}
%fix algorithmusverzeichnis
\floatname{algorithm}{Algorithmus}
\captionsetup[algorithm]{name=Algorithmus,labelsep=colon}
%-----------------------------------
% sauber formatierter Quelltext
%-----------------------------------
\usepackage{listings}

% JavaScript als Sprache definieren:
\lstdefinelanguage{JavaScript}{
	keywords={break, super, case, extends, switch, catch, finally, for, const, function, try, continue, if, typeof, debugger, var, default, in, void, delete, instanceof, while, do, new, with, else, return, yield, enum, let, await},
	keywordstyle=\color{blue}\bfseries,
	ndkeywords={class, export, boolean, throw, implements, import, this, interface, package, private, protected, public, static},
	ndkeywordstyle=\color{darkgray}\bfseries,
	identifierstyle=\color{black},
	sensitive=false,
	comment=[l]{//},
	morecomment=[s]{/*}{*/},
	commentstyle=\color{purple}\ttfamily,
	stringstyle=\color{red}\ttfamily,
	morestring=[b]',
	morestring=[b]"
}

\lstset{
	%language=JavaScript,
	numbers=left,
	numberstyle=\tiny,
	numbersep=5pt,
	breaklines=true,
	showstringspaces=false,
	frame=l ,
	xleftmargin=5pt,
	xrightmargin=5pt,
	basicstyle=\ttfamily\scriptsize,
	stepnumber=1,
	keywordstyle=\color{blue},          % keyword style
  	commentstyle=\color{dkgreen},       % comment style
  	stringstyle=\color{mauve}         % string literal style
}

\lstdefinelanguage{CSharp}{
  keywords={
    abstract, as, base, bool, break, byte, case, catch, char, checked, class, const, continue, 
    decimal, default, delegate, do, double, else, enum, event, explicit, extern, false, finally, 
    fixed, float, for, foreach, goto, if, implicit, in, int, interface, internal, is, lock, long, 
    namespace, new, null, object, operator, out, override, params, private, protected, public, 
    readonly, ref, return, sbyte, sealed, short, sizeof, stackalloc, static, string, struct, 
    switch, this, throw, true, try, typeof, uint, ulong, unchecked, unsafe, ushort, using, 
    virtual, void, volatile, while
  },
  keywordstyle=\color{lightblue}\bfseries,
  identifierstyle=\color{black},
  sensitive=true,
  comment=[l]{//},
  morecomment=[s]{/*}{*/},
  commentstyle=\color{gray}\ttfamily,
  stringstyle=\color{dkgreen}\ttfamily,
  morestring=[b]',
  morestring=[b]",
  classoffset=0,
  classoffset=1,
  morekeywords={
    GameObject, Transform, Vector2, Vector3, Vector2Int, Vector3Int, Quaternion, MonoBehaviour, 
    Rigidbody, Collider, BoxCollider, CapsuleCollider, Component, Camera, Time, Input, Physics, 
    RaycastHit, Ray, Mathf, Coroutine, Canvas, CanvasScaler, RectTransform, Button, Text, Image, 
    GridLayoutGroup, WaitForSeconds, StartCoroutine, Instantiate, Destroy, Cinemachine, Tilemap, 
    Grid, IEnumerator, Dictionary, List, SceneManager, PlayerPrefs
  },
  keywordstyle=\color{darkorange}\bfseries,
  classoffset=0,
  classoffset=2,
  morekeywords={private, public, protected, readonly, static, abstract},
  keywordstyle=\color{mauve}\bfseries,
  classoffset=0,
  classoffset=3,
  morekeywords={
    Awake, Start, Update, LateUpdate, FixedUpdate, OnEnable, OnDisable, OnDestroy, OnTriggerEnter, 
    OnTriggerStay, OnTriggerExit, OnCollisionEnter, OnCollisionStay, OnCollisionExit, OnMouseDown, 
    OnMouseUp, GetComponent, GetComponentInChildren, GetComponentInParent, AddComponent, FindObjectOfType, 
    FindObjectsOfType, Instantiate, Destroy, DontDestroyOnLoad, Invoke, InvokeRepeating, CancelInvoke, 
    IsInvoking, StartCoroutine, StopCoroutine, StopAllCoroutines, SetActive, gameObject, transform, 
    position, rotation, localPosition, localRotation, localScale, parent, childCount, GetChild,
    ScreenPointToRay, Raycast, Clamp, FloorToInt, CeilToInt, RoundToInt, Abs, Sin, Cos, Tan, 
    Asin, Acos, Atan, Atan2, Sqrt, Pow, Exp, Log, Log10, Ceil, Floor, Round, Sign, Min, Max, 
    Approximately, Lerp, LerpUnclamped, SmoothStep, MoveTowards, deltaTime, fixedDeltaTime, 
    GetAxis, GetButton, GetButtonDown, GetButtonUp, GetKey, GetKeyDown, GetKeyUp, GetMouseButton, 
    GetMouseButtonDown, GetMouseButtonUp, mousePosition, PlaceBlocks, SnapToGrid
  },
  keywordstyle=\color{marineblue}\bfseries,
  classoffset=0,
  classoffset=4,
  morekeywords={
    var, dynamic, get, set, value, partial, yield, add, remove, alias, ascending, descending, 
    from, group, into, join, let, orderby, select, where, async, await, nameof, when
  },
  keywordstyle=\color{darkyellow}\bfseries,
  classoffset=0
}

\lstset{
    language=CSharp,
    extendedchars=true,
    basicstyle=\footnotesize\ttfamily,
    showstringspaces=false,
    showspaces=false,
    numbers=left,
    numberstyle=\footnotesize,
    numbersep=9pt,
    tabsize=2,
    breaklines=true,
    showtabs=false,
    captionpos=b
  }

\lstdefinelanguage{TypeScript}{
  keywords={
    break, case, catch, class, continue, const, constructor, debugger, default, delete, do, else, enum, export, extends, false, finally, for, from, function, if, import, in, instanceof, let, new, null, return, super, switch, this, throw, true, try, typeof, var, void, while, with, as, any, implements, interface, package, static, yield, number, string, boolean, symbol, abstract, async, await, get, set, readonly, id, never, type, namespace, module, declare, global, keyof, unknown, infer, is, satisfies, override, unique, of
  },
  keywordstyle=\color{lightblue}\bfseries,
  identifierstyle=\color{black},
  sensitive=true,
  comment=[l]{//},
  morecomment=[s]{/*}{*/},
  commentstyle=\color{gray}\ttfamily,
  stringstyle=\color{dkgreen}\ttfamily,
  morestring=[b]',
  morestring=[b]",
  classoffset=0,
  classoffset=1,
  morekeywords={
    Component, FormBuilder, FormGroup, Validators, AbilityService, AbilityEntity, AbilityController, 
    MonsterEntity, Skills, TimelineNodeBase, Injectable, Repository, undefined, Ability, Entity, 
    PrimaryGeneratedColumn, Column, InjectRepository, OnInit, OnDestroy, AfterViewInit, 
    ChangeDetectorRef, ElementRef, EventEmitter, ViewContainerRef, TemplateRef, 
    ComponentFactoryResolver, NgZone, Router, ActivatedRoute, HttpClient, FormControl, 
    FormArray, Observable, Subscription, Subject, BehaviorSubject, ReplaySubject, 
    ChangeDetectionStrategy, ViewEncapsulation, Input, Output, Directive, HostBinding, 
    HostListener, Pipe, PipeTransform, Injector, NgModule, Controller, Get, Post, Param, Body, Delete, Put, AbilityEndpoint, Creature, Monster
  },
  keywordstyle=\color{darkorange}\bfseries,
  classoffset=0,
  classoffset=2,
  morekeywords={private, public, protected, readonly, static, abstract},
  keywordstyle=\color{mauve}\bfseries,
  classoffset=0,
  classoffset=3,
  morekeywords={
    createForm, group, array, createAttributesFormGroup, createSkillsFormGroup, findAll, 
    findOne, create, update, remove, constructor, find, save, findOneBy, ngOnInit, 
    ngOnDestroy, ngAfterViewInit, subscribe, unsubscribe, emit, next, complete, error, 
    pipe, map, filter, tap, switchMap, mergeMap, concatMap, takeUntil, takeWhile, 
    distinctUntilChanged, debounceTime, throttleTime, catchError, finalize, forEach, 
    push, pop, shift, unshift, splice, slice, join, split, concat, includes, indexOf
  },
  keywordstyle=\color{marineblue}\bfseries,
  classoffset=0,
  classoffset=4,
  morekeywords={
    @Input, @Output, @Component, @Injectable, @Directive, @Pipe, @NgModule, @Entity, 
    @campaign-manager/shared, @PrimaryGeneratedColumn, @Column, @nestjs/common, 
    @nestjs/typeorm, @InjectRepository, @Get, @Param, @Post, @Put, @Delete, 
    @ManyToMany, @JoinTable, @OneToMany, @ManyToOne, @OneToOne, @JoinColumn, @ViewChild, 
    @ViewChildren, @HostBinding, @HostListener, @ContentChild, @ContentChildren, 
    @Optional, @Inject, @Self, @SkipSelf, @Host, @Query, @Body, @Headers, @Req, @Res
  },
  keywordstyle=\color{darkyellow}\bfseries,
  classoffset=0
}

\lstset{
  language=TypeScript,
  extendedchars=true,
  basicstyle=\footnotesize\ttfamily,
  showstringspaces=false,
  showspaces=false,
  numbers=left,
  numberstyle=\footnotesize,
  numbersep=9pt,
  tabsize=2,
  breaklines=true,
  showtabs=false,
  captionpos=b
}

\lstdefinelanguage{HTML}{
  sensitive=false,
  morecomment=[s]{<!--}{-->},
  commentstyle=\color{gray}\ttfamily,
  morestring=[b]',
  morestring=[b]",
  stringstyle=\color{dkgreen}\ttfamily,
  keywords={html,head,title,base,link,meta,style,script,body,section,nav,article,aside,h1,h2,h3,h4,h5,h6,header,footer,address,main,p,hr,pre,blockquote,ol,ul,li,dl,dt,dd,figure,figcaption,div,a,em,strong,small,s,cite,dfn,abbr,time,code,var,samp,kbd,sub,sup,i,b,u,mark,ruby,rt,rp,bdi,bdo,span,br,wbr,ins,del,image,img,iframe,embed,object,param,video,audio,source,track,canvas,map,area,svg,math,table,caption,colgroup,col,tbody,thead,tfoot,tr,td,th,form,fieldset,legend,label,input,button,select,datalist,optgroup,option,textarea,output,progress,meter,details,summary,menu,menuitem,command,device,element,shadow,template,decorator,font,center,big,blink,marquee,p,-,floatLabel,cm,data,panel},
  keywordstyle=\color{lightblue}\bfseries,
  ndkeywords={class,id,style,href,src,alt,title,type,value,method,placeholder,charset,content,http-equiv,rel,sizes,pInputText,for,formControlName},
  ndkeywordstyle=\color{mauve}\bfseries,
}

\lstset{
  language=HTML,
  extendedchars=true,
  basicstyle=\footnotesize\ttfamily,
  showstringspaces=false,
  showspaces=false,
  numbers=left,
  numberstyle=\footnotesize,
  numbersep=9pt,
  tabsize=2,
  breaklines=true,
  showtabs=false,
  captionpos=b
}

%-----------------------------------
%Literaturverzeichnis Einstellungen
%-----------------------------------

% Biblatex

\usepackage{url}
\urlstyle{same}

%%%% Neuer Leitfaden (2018)
\usepackage[
backend=biber,
style=ext-authoryear-ibid, % Auskommentieren und nächste Zeile einkommentieren, falls "Ebd." (ebenda) nicht für sich-wiederholende Fussnoten genutzt werden soll.
%style=ext-authoryear,
maxcitenames=3,	% mindestens 3 Namen ausgeben bevor et. al. kommt
maxbibnames=999,
mergedate=false,
date=iso,
seconds=true, %werden nicht verwendet, so werden aber Warnungen unterdrückt.
urldate=iso,
innamebeforetitle,
dashed=false,
autocite=footnote,
doi=false,
useprefix=true, % 'von' im Namen beachten (beim Anzeigen)
mincrossrefs = 1
]{biblatex}%iso dateformat für YYYY-MM-DD

%weitere Anpassungen für BibLaTex
\input{skripte/modsBiblatex2018}

%%%%% Alter Leitfaden. Ggf. Einkommentieren und Bereich hierüber auskommentieren
%\usepackage[
%backend=biber,
%style=numeric,
%citestyle=authoryear,
%url=false,
%isbn=false,
%notetype=footonly,
%hyperref=false,
%sortlocale=de]{biblatex}

%weitere Anpassungen für BibLaTex
%\input{skripte/modsBiblatex}

%%%% Ende Alter Leitfaden

%Bib-Datei einbinden
\addbibresource{literatur/literatur.bib}

% Zeilenabstand im Literaturverzeichnis ist Einzeilig
% siehe Leitfaden S. 14
\AtBeginBibliography{\singlespacing}

%-----------------------------------
% Silbentrennung
%-----------------------------------
\usepackage{hyphsubst}
\HyphSubstIfExists{ngerman-x-latest}{%
\HyphSubstLet{ngerman}{ngerman-x-latest}}{}

%-----------------------------------
% Pfad fuer Abbildungen
%-----------------------------------
\graphicspath{{./}{./abbildungen/}}

%-----------------------------------
% Weitere Ebene einfügen
%-----------------------------------
\input{skripte/weitereEbene}

%-----------------------------------
% Paket für die Nutzung von Anhängen
%-----------------------------------
\usepackage{appendix}

%-----------------------------------
% Zeilenabstand 1,5-zeilig
%-----------------------------------
\usepackage{setspace}
\onehalfspacing

%-----------------------------------
% Absätze durch eine neue Zeile
%-----------------------------------
\setlength{\parindent}{0mm}
\setlength{\parskip}{0.8em plus 0.5em minus 0.3em}

\sloppy					%Abstände variieren
\pagestyle{headings}

%----------------------------------
% Präfix in das Abbildungs- und Tabellenverzeichnis aufnehmen, statt nur der Nummerierung (siehe Issue #206).
%----------------------------------
\KOMAoption{listof}{entryprefix} % Siehe KOMA-Script Doku v3.28 S.153
\BeforeStartingTOC[lof]{\renewcommand*\autodot{:}} % Für den Doppelpunkt hinter Präfix im Abbildungsverzeichnis
\BeforeStartingTOC[lot]{\renewcommand*\autodot{:}} % Für den Doppelpunkt hinter Präfix im Tabellenverzeichnis
\BeforeStartingTOC[loa]{\renewcommand*\autodot{:}} % Für den Doppelpunkt hinter Präfix im Algorithmenverzeichnis

%-----------------------------------
% Abkürzungsverzeichnis
%-----------------------------------
\usepackage[printonlyused]{acronym}

%-----------------------------------
% Symbolverzeichnis
%-----------------------------------
% Quelle: https://www.namsu.de/Extra/pakete/Listofsymbols.pdf
\usepackage[final]{listofsymbols}

%-----------------------------------
% Glossar
%-----------------------------------
\usepackage{glossaries}
\glstoctrue %Auskommentieren, damit das Glossar nicht im Inhaltsverzeichnis angezeigt wird.
\makenoidxglossaries
\input{abkuerzungen/glossar}

%-----------------------------------
% PDF Meta Daten setzen
%-----------------------------------
\usepackage[hyperfootnotes=false]{hyperref} %hyperfootnotes=false deaktiviert die Verlinkung der Fußnote. Ansonsten inkompaibel zum Paket "footmisc"
% Behebt die falsche Darstellung der Lesezeichen in PDF-Dateien, welche eine Übersetzung besitzen
% siehe Issue 149
\makeatletter
\pdfstringdefDisableCommands{\let\selectlanguage\@gobble}
\makeatother

\hypersetup{
    pdfinfo={
        Title={\myTitel},
        Subject={\myStudiengang},
        Author={\myAutor},
        Build=1.1
    }
}

%-----------------------------------
% PlantUML
%-----------------------------------
%\usepackage{plantuml}

%-----------------------------------
% Umlaute in Code korrekt darstellen
% siehe auch: https://en.wikibooks.org/wiki/LaTeX/Source_Code_Listings
%-----------------------------------
\lstset{literate=
	{á}{{\'a}}1 {é}{{\'e}}1 {í}{{\'i}}1 {ó}{{\'o}}1 {ú}{{\'u}}1
	{Á}{{\'A}}1 {É}{{\'E}}1 {Í}{{\'I}}1 {Ó}{{\'O}}1 {Ú}{{\'U}}1
	{à}{{\`a}}1 {è}{{\`e}}1 {ì}{{\`i}}1 {ò}{{\`o}}1 {ù}{{\`u}}1
	{À}{{\`A}}1 {È}{{\'E}}1 {Ì}{{\`I}}1 {Ò}{{\`O}}1 {Ù}{{\`U}}1
	{ä}{{\"a}}1 {ë}{{\"e}}1 {ï}{{\"i}}1 {ö}{{\"o}}1 {ü}{{\"u}}1
	{Ä}{{\"A}}1 {Ë}{{\"E}}1 {Ï}{{\"I}}1 {Ö}{{\"O}}1 {Ü}{{\"U}}1
	{â}{{\^a}}1 {ê}{{\^e}}1 {î}{{\^i}}1 {ô}{{\^o}}1 {û}{{\^u}}1
	{Â}{{\^A}}1 {Ê}{{\^E}}1 {Î}{{\^I}}1 {Ô}{{\^O}}1 {Û}{{\^U}}1
	{œ}{{\oe}}1 {Œ}{{\OE}}1 {æ}{{\ae}}1 {Æ}{{\AE}}1 {ß}{{\ss}}1
	{ű}{{\H{u}}}1 {Ű}{{\H{U}}}1 {ő}{{\H{o}}}1 {Ő}{{\H{O}}}1
	{ç}{{\c c}}1 {Ç}{{\c C}}1 {ø}{{\o}}1 {å}{{\r a}}1 {Å}{{\r A}}1
	{€}{{\EUR}}1 {£}{{\pounds}}1 {„}{{\glqq{}}}1
}

%-----------------------------------
% Kopfbereich / Header definieren
%-----------------------------------
\pagestyle{fancy}
\fancyhf{}
% Seitenzahl oben, mittig, mit Strichen beidseits
% \fancyhead[C]{-\ \thepage\ -}

% Seitenzahl oben, mittig, entsprechend Leitfaden ohne Striche beidseits
\fancyhead[C]{\thepage}
%\fancyhead[L]{\leftmark}							% kein Footer vorhanden
% Waagerechte Linie unterhalb des Kopfbereiches anzeigen. Laut Leitfaden ist
% diese Linie nicht erforderlich. Ihre Breite kann daher auf 0pt gesetzt werden.
\renewcommand{\headrulewidth}{0.4pt}
%\renewcommand{\headrulewidth}{0pt}

%-----------------------------------
% Damit die hochgestellten Zahlen auch auf die Fußnote verlinkt sind (siehe Issue 169)
%-----------------------------------
\hypersetup{colorlinks=true, breaklinks=true, linkcolor=darkblack, citecolor=darkblack, menucolor=darkblack, urlcolor=darkblack, linktoc=all, bookmarksnumbered=false, pdfpagemode=UseOutlines, pdftoolbar=true}
\urlstyle{same}%gleiche Schriftart für den Link wie für den Text

%-----------------------------------
% Start the document here:
%-----------------------------------
\begin{document}

\pagenumbering{Roman}								% Seitennumerierung auf römisch umstellen
\newcolumntype{C}{>{\centering\arraybackslash}X}	% Neuer Tabellen-Spalten-Typ:
%Zentriert und umbrechbar

%-----------------------------------
% Textcommands
%-----------------------------------
\input{skripte/textcommands}

%-----------------------------------
% Titlepage
%-----------------------------------
\input{kapitel/titelseite}

%-----------------------------------
% Vorwort (optional; bei Verwendung beide Zeilen entkommentieren und unter Inhaltsverzeichnis setcounter entsprechend anpassen)
%-----------------------------------
%\input{kapitel/vorwort/vorwort}
%\newpage

%-----------------------------------
% Inhaltsverzeichnis
%-----------------------------------
% Um das Tabellen- und Abbbildungsverzeichnis zu de/aktivieren ganz oben in Documentclass schauen
\setcounter{page}{2}
\addtocontents{toc}{\protect\enlargethispage{-20mm}}% Die Zeile sorgt dafür, dass das Inhaltsverzeichnisseite auf die zweite Seite gestreckt wird und somit schick aussieht. Das sollte eigentlich automatisch funktionieren. Wer rausfindet wie, kann das gern ändern.
\setcounter{tocdepth}{4}
\tableofcontents
\newpage

%-----------------------------------
% Abbildungsverzeichnis
%-----------------------------------
\listoffigures
\newpage
%-----------------------------------
% Tabellenverzeichnis - pp2 keine Tabellen vorhanden.
%-----------------------------------
\listoftables
\newpage
%-----------------------------------
% Algorithmenverzeichnis
%-----------------------------------

\newcommand{\algorithmlistnames}{Algorithmenverzeichnis}

\newcommand\entrywithprefixformat[1]{
	\algorithmname~#1
}

\DeclareTOCStyleEntry[
	level=1,
	indent=1.5em,
	numwidth=2.3em,
	entrynumberformat=\entrywithprefixformat,
	dynnumwidth
]{tocline}{algorithm}

\renewcommand{\listofalgorithms}{
	\cleardoublepage
	\listoftoc{loa}
}
\newcommand{\listofloaname}{Algorithmenverzeichnis}
\listofalgorithms
\newpage
%-----------------------------------
% Abkürzungsverzeichnis
%-----------------------------------
% Falls das Abkürzungsverzeichnis nicht im Inhaltsverzeichnis angezeigt werden soll
% dann folgende Zeile auskommentieren.
% \addcontentsline{toc}{section}{\abbreHeadingName}

\section*{\langde{Abkürzungsverzeichnis}\langen{List of Abbreviations}}

\begin{acronym}[NPC]\itemsep0pt %der Parameter in Klammern sollte die längste Abkürzung sein. Damit wird der Abstand zwischen Abkürzung und Übersetzung festgelegt
  \acro{NPC}{Non-Player-Character}
\end{acronym}
\newpage

%-----------------------------------
% Symbolverzeichnis
%-----------------------------------
% In Overleaf führt der Einsatz des Symbolverzeichnisses zu einem Fehler, der aber ignoriert werdne kann
% Falls das Symbolverzeichnis nicht im Inhaltsverzeichnis angezeigt werden soll
% dann folgende Zeile auskommentieren.
%\addcontentsline{toc}{section}{\symheadingname}
%\input{skripte/symbolDef}
%\listofsymbols
%\newpage

%-----------------------------------
% Glossar
%-----------------------------------
\printnoidxglossaries
\newpage

%-----------------------------------
% Sperrvermerk
%-----------------------------------
%\input{kapitel/anhang/sperrvermerk}

%-----------------------------------
% Seitennummerierung auf arabisch und ab 1 beginnend umstellen
%-----------------------------------
\pagenumbering{arabic}
\setcounter{page}{1}

%-----------------------------------
% Kapitel / Inhalte
%-----------------------------------
% Die Kapitel werden über folgende Datei eingebunden
% Hinzugefügt aufgrund von Issue 167
%-----------------------------------
% Kapitel / Inhalte
%-----------------------------------
\newpage
\section{Einleitung}

\subsection{Problemstellung}

\subsection{Zielsetzung}

\subsection{Gang der Untersuchung}

\newpage
\section{Grundlagen}

\subsection{Animation Controller und Blender} % 500 M
Mape uwu
\subsection{UI AND UX}  % 500 S
Sucki uwu
\subsection{Animation-Rigging} % 500 T
Mit dem Animation-Rigging-Paket von Unity können prozedurale Animationen erstellt und bearbeitet werden, die sich dynamisch zur Laufzeit oder während einer Animation verändern lassen. Dadurch sind flexible Bewegungen innerhalb einer Animation sowie interaktive Animationen mit anderen Entitäten möglich. Dadurch werden die Funktionen des von Anfang an eingebauten Animation-Systems erweitert und verbessert. Das Paket wird von Unity selbst entwickelt und ist aktuell in der Version 1.3.0 verfügbar. Es lässt sich wie andere Pakete über den Unity Package Manager installieren. \footcite[\vglf][]{unity.animationRigging}
Interaktive Animationen sind in vielen Spielen ein wichtiger Bestandteil für ein immersives und dynamisches Spielerlebnis. Sie ermöglichen es beispielsweise, dass Spielcharaktere realistisch mit der Umgebung interagieren, indem sie Objekte aus verschiedenen Winkeln aufheben, ohne dass exakt vordefinierte Animationen verwendet werden müssen. In dem Videospiel Half-Life 2 wird dies zum Beispiel genutzt, um den Kopf eines \ac{NPC}, immer in Richtung des Spielers zu drehen und so eine realistischere Interaktion zu ermöglichen. In der \autoref{fig:hl2_animation_rigging} ist ein Beispiel für eine solche Animation zu sehen.

\begin{figure}[H]
  \caption[Dynamische Drehung des Kopfes]{Dynamische Drehung des Kopfes}\label{fig:hl2_animation_rigging}
  \includegraphics[width=1.0\textwidth]{hl2_animation_rigging.png}
\end{figure}

Als Rigging bezeichnet man das Erstellen einer Struktur, die für die Animation von Charakteren oder Objekten verwendet wird. Diese Struktur besteht aus mehreren miteinander verbundenen Knochenelementen, die ein Skelett, auch Rig genannt, bilden. Mithilfe des Skeletts können anschließend die einzelnen Teile des Charakters oder Objekts animiert werden. Die Transformationen der Knochenelemente beeinflussen das Mesh des Modells, sodass sich das gesamte Objekt entsprechend der Animation bewegt. \footcite[\vglf][]{riggingBasics} \footcite[\vglf][]{characterRiggingForGames}
Das Unity-Animation-Rigging-Paket umfasst neben dem Rigging selbst auch constraint-basierte Steuerungen und prozedurale Anpassungen. Dies ermöglicht einerseits das gezielte Beeinflussen bestimmter Teile des Skeletts, zum Beispiel, um Objekte vom Boden aufzuheben oder den Kopf nach einem Objekt auszurichten, und andererseits die Übersteuerung von Animationen, um Überblendungen zwischen Animationen zu ermöglichen oder Animationen dynamisch an bestimmte Situationen anzupassen. \footcite[\vglf][]{unity.animationRigging}
Constraints sind Regeln, die definieren, wie sich die Teile beziehungsweise die Knochenelemente verhalten sollen. Der Multi-Aim-Constraint richtet beispielsweise mehrere Knochenelemente auf ein ausgewähltes Ziel aus. Ein Rig kann aus mehreren Constraints bestehen und besitzt somit eine Sammlung von Constraints beziehungsweise Regeln. Die Rig-Layer erlauben es, verschiedene Rigs mit Priorisierung zu kombinieren. Für die einzelnen Rigs und Constraints eines Rigs können Gewichtungen, auch Weights genannt, festgelegt werden, um anzugeben, wie stark die einzelnen Constraints auf das Rig und das Rig auf das Mesh wirken sollen.
In Unity können Rigs mithilfe der Rig-Builder-Komponente erstellt werden. Diese verwaltet alle Rigs eines GameObjects beziehungsweise 3D-Modells und ermöglicht es, diese zu kombinieren und zu priorisieren. Die Rigs selbst werden über die Rig-Komponente erstellt und anschließend in der Rig-Builder-Komponente hinzugefügt. Ein Rig besitzt eine Kopie aller Knochenelemente des 3D-Modells sowie eine Eingabe zur Angabe der Gewichtung des gesamten Rigs. Die einzelnen Constraints werden als weitere Komponenten dem Rig hinzugefügt. Dafür gibt es eine Auswahl an verschiedenen Komponenten, die jeweils für einen bestimmten Constraint-Typ zuständig sind. Zudem muss für jeden Constraint mindestens ein Knochenelement angegeben werden, das beeinflusst werden soll. Wie das Rig besitzt jeder Constraint ebenfalls eine Eingabe für die Gewichtung des Constraints. Diese gibt an, wie stark der Constraint auf das Rig und damit auf das Mesh wirkt. \footcite[\vglf][]{unity.animationRiggingManual}






\newpage

\section{Weiterentwicklung des Spiels}

\subsection{Boss-Gegner}  % 2000 T 
Temu uwu
\subsection{Module System}  % 1000 S

\subsection{3D-Modelle und Animationen}   % 1000 M
\subsection{UI} % 1000 S
test







\newpage
\section{Fazit}

\subsection{Zusammenfassung der Ergebnisse} 

\subsection{Lessons Learned}

\subsection{Ausblick}






%-----------------------------------
% Apendix / Anhang
%-----------------------------------

%-----------------------------------
% Literaturverzeichnis
%-----------------------------------
\newpage

% Die folgende Zeile trägt ALLE Werke aus literatur.bib in das
% Literaturverzeichnis ein, egal ob sie zietiert wurden oder nicht.
% Der Befehl ist also nur zum Test der Skripte sinnvoll und muss bei echten
% Arbeiten entfernt werden.
%\nocite{*}

%\addcontentsline{toc}{section}{Literatur}

% Die folgenden beiden Befehle würden ab dem Literaturverzeichnis wieder eine
% römische Seitennummerierung nutzen.
% Das ist nach dem Leitfaden nicht zu tun. Dort steht nur dass 'sämtliche
% Verzeichnisse VOR dem Textteil' römisch zu nummerieren sind. (vgl. S. 3)
%\pagenumbering{Roman} %Zähler wieder römisch ausgeben
%\setcounter{page}{4}  %Zähler manuell hochsetzen

% Ausgabe des Literaturverzeichnisses

% Keine Trennung der Werke im Literaturverzeichnis nach ihrer Art
% (Online/nicht-Online)
%\begin{RaggedRight}
%\printbibliography
%\end{RaggedRight}

% Alternative Darstellung, die laut Leitfaden genutzt werden sollte.
% Dazu die Zeilen auskommentieren und folgenden code verwenden:

% Literaturverzeichnis getrennt nach Nicht-Online-Werken und Online-Werken
% (Internetquellen).
% Die Option nottype=online nimmt alles, was kein Online-Werk ist.
% Die Option heading=bibintoc sorgt dafür, dass das Literaturverzeichnis im
% Inhaltsverzeichnis steht.
% Es ist übrigens auch möglich mehrere type- bzw. nottype-Optionen anzugeben, um
% noch weitere Arten von Zusammenfassungen eines Literaturverzeichnisse zu
% erzeugen.
% Beispiel: [type=book,type=article]
\printbibliography[nottype=online,heading=bibintoc,title={\langde{Literaturverzeichnis}\langen{Bibliography}}]

% neue Seite für Internetquellen-Verzeichnis
\newpage

% Laut Leitfaden 2018, S. 14, Fussnote 44 stehen die Internetquellen NICHT im
% Inhaltsverzeichnis, sondern gehören zum Literaturverzeichnis.
% Die Option heading=bibintoc würde die Internetquelle als eigenen Eintrag im
% Inhaltsverzeicnis anzeigen.
%\printbibliography[type=online,heading=bibintoc,title={\headingNameInternetSources}]
\printbibliography[type=online,heading=subbibliography,title={\headingNameInternetSources}]

\input{kapitel/anhang/erklaerung_timo.tex}
\input{kapitel/anhang/erklaerung_simon.tex}
\input{kapitel/anhang/erklaerung_mario.tex}

\end{document}
